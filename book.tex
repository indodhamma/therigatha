%% LyX 2.3.7 created this file.  For more info, see http://www.lyx.org/.
%% Do not edit unless you really know what you are doing.
\documentclass[b5paper]{book}
\usepackage{lmodern}
\usepackage{lmodern}
\usepackage[T1]{fontenc}
\usepackage[utf8]{inputenc}
\usepackage{fancyhdr}
\pagestyle{fancy}
\setcounter{secnumdepth}{-1}
\setcounter{tocdepth}{3}
\usepackage{amsmath}
\usepackage{amssymb}
\usepackage[unicode=true,
 bookmarks=false,
 breaklinks=false,pdfborder={0 0 1},backref=section,colorlinks=false]
 {hyperref}
\hypersetup{
 hidelinks,pdfcreator={LaTeX via pandoc}}

\makeatletter

%%%%%%%%%%%%%%%%%%%%%%%%%%%%%% LyX specific LaTeX commands.
\pdfpageheight\paperheight
\pdfpagewidth\paperwidth


\@ifundefined{date}{}{\date{}}
%%%%%%%%%%%%%%%%%%%%%%%%%%%%%% User specified LaTeX commands.
% Options for packages loaded elsewhere
\usepackage{titlesec}
\titleformat{\chapter}[hang]{\Huge\bfseries}{Chapter~\thechapter}{.5em}{}

%
\usepackage{iftex}
\ifPDFTeX
  \usepackage{textcomp}% provide euro and other symbols
\else % if luatex or xetex
  \usepackage{unicode-math}% this also loads fontspec
  \defaultfontfeatures{Scale=MatchLowercase}
  \defaultfontfeatures[\rmfamily]{Ligatures=TeX,Scale=1}
\fi
\ifPDFTeX\else
  % xetex/luatex font selection
\fi
% Use upquote if available, for straight quotes in verbatim environments
\IfFileExists{upquote.sty}{\usepackage{upquote}}{}
\IfFileExists{microtype.sty}{% use microtype if available
  \usepackage[]{microtype}
  \UseMicrotypeSet[protrusion]{basicmath} % disable protrusion for tt fonts
}{}

\@ifundefined{KOMAClassName}{% if non-KOMA class
  \IfFileExists{parskip.sty}{%
    \usepackage{parskip}
  }{% else
    \setlength{\parindent}{0pt}
    \setlength{\parskip}{6pt plus 2pt minus 1pt}}
}{% if KOMA class
  \KOMAoptions{parskip=half}}

\usepackage{xcolor}
\setlength{\emergencystretch}{3em} % prevent overfull lines
\providecommand{\tightlist}{%
  \setlength{\itemsep}{0pt}\setlength{\parskip}{0pt}}
 % remove section numbering
\ifLuaTeX
  \usepackage{selnolig}% disable illegal ligatures
\fi
\IfFileExists{bookmark.sty}{\usepackage{bookmark}}{\usepackage{hyperref}}
\IfFileExists{xurl.sty}{\usepackage{xurl}}{} % add URL line breaks if available
\urlstyle{same}

% Let's hide the credits from ToC
\newcounter{oldtocdepth}

\newcommand{\hidefromtoc}{%
  \setcounter{oldtocdepth}{\value{tocdepth}}%
  \addtocontents{toc}{\protect\setcounter{tocdepth}{-10}}%
}

\newcommand{\unhidefromtoc}{%
  \addtocontents{toc}{\protect\setcounter{tocdepth}{\value{oldtocdepth}}}%
}

% Let's set up our header settings
\usepackage{fancyhdr}
\fancyhf{}
\fancyhead[LE,RO]{\thepage}
\fancyhead[LO]{Therigathi}
\fancyhead[ER]{\leftmark}
\fancypagestyle{plain}{%
  \fancyhf{}%
  \renewcommand{\headrulewidth}{0pt}% Line at the header invisible
  \renewcommand{\footrulewidth}{0pt}% Line at the footer visible
}
\author{}

\makeatother

\begin{document}
\tableofcontents{}

\chapter{Kelompok Satu}

\hypertarget{bhikkhunux12b-tanpa-nama-1}{

\section{1.1 Bhikkhunī Tanpa Nama (1)}

\label{bhikkhunux12b-tanpa-nama-1}}
\begin{quote}
Tidurlah dengan damai, bhikkhunī kecil,\\
 Terbalut jubah yang engkau jahit sendiri;\\
 Karena keinginanmu telah dipadamkan,\\
 Bagai sayuran rebus mengering dalam panci. 
\end{quote}
Demikianlah syair ini dilafalkan oleh seorang bhikkhunī tertentu yang
tidak diketahui namanya.

\hypertarget{muttux101-1}{

\section{1.2 Muttā (1)}

\label{muttux101-1}}
\begin{quote}
Muttā, terbebaslah dari belenggumu,\\
 Bagaikan bulan yang terbebas dari cengkeraman Rāhu, sang gerhana.\\
 Ketika batinmu terbebaskan,\\
 Nikmatilah makananmu yang bebas dari utang. 
\end{quote}
Demikianlah Sang Buddha secara rutin menasihati calon bhikkhunī Muttā
melalui syair ini.

\hypertarget{puux1e47ux1e47ux101}{

\newpage

\section{1.3 Pu\d{n}\d{n}ā}

\label{puux1e47ux1e47ux101}}
\begin{quote}
Pu\d{n}\d{n}ā, penuhilah dengan kualitas-kualitas baik,\\
 Bagaikan bulan pada tanggal lima belas,\\
 Ketika kebijaksanaanmu penuh,\\
 hancurkanlah kumpulan kegelapan. 
\end{quote}
Demikianlah syair ini dilafalkan oleh bhikkhunī senior Pu\d{n}\d{n}ā.

\hypertarget{tissux101}{

\section{1.4 Tissā}

\label{tissux101}}
\begin{quote}
Tissā, berlatihlah dalam latihan-latihan---\\
 Jangan biarkan latihan berlalu.\\
 Terlepas dari segala kemelekatan,\\
 Hiduplah dalam dunia yang bebas dari kekotoran. 
\end{quote}
\hypertarget{tissux101-yang-lain}{

\section{1.5 Tissā yang lain}

\label{tissux101-yang-lain}}
\begin{quote}
Tissā, kerahkan dirimu untuk mendapatkan kualitas-kualitas baik---\\
 Jangan biarkan momen ini berlalu.\\
 Karena jika engkau kehilangan momenmu,\\
 Engkau akan bersedih ketika engkau dikirim ke neraka. 
\end{quote}
\hypertarget{dhux12brux101}{

\section{1.6 Dhīrā}

\label{dhux12brux101}}
\begin{quote}
Dhīrā, sentuhlah lenyapnya,\\
 Diamnya persepsi yang membahagiakan.\\
 Menangkanlah pemadaman,\\
 Suaka tertinggi. 
\end{quote}
\hypertarget{vux12brux101}{

\section{1.7 Vīrā}

\label{vux12brux101}}
\begin{quote}
Ia dikenal sebagai Vīrā karena kualitas-kualitas keberaniannya,\\
 Seorang bhikkhunī dengan indria-indria terkembang.\\
 Ia membawa jasmani terakhirnya,\\
 Setelah menumpas Māra dan tunggangannya. 
\end{quote}
\hypertarget{mittux101-1}{

\section{1.8 Mittā (1)}

\label{mittux101-1}}
\begin{quote}
Setelah meninggalkan keduniawian karena keyakinan,\\
 Hargailah teman-teman spiritualmu, Mittā.\\
 Kembangkanlah kualitas-kualitas terampilmu\\
 Demi menemukan suaka. 
\end{quote}
\hypertarget{bhadrux101}{

\section{1.9 Bhadrā}

\label{bhadrux101}}
\begin{quote}
Setelah meninggalkan keduniawian karena keyakinan,\\
 Hargailah berkah-berkahmu, Bhadrā.\\
 Kembangkanlah kualitas-kualitas terampil\\
 Demi suaka tertinggi. 
\end{quote}
\hypertarget{upasamux101}{

\section{1.10 Upasamā}

\label{upasamux101}}
\begin{quote}
Upasamā, seberangilah banjir,\\
 Wilayah kematian yang sulit dilewati.\\
 Setelah engkau menumpas Māra dan tunggangannya,\\
 Bawalah jasmani terakhirmu. 
\end{quote}
\hypertarget{muttux101-2}{

\section{1.11 Muttā (2)}

\label{muttux101-2}}
\begin{quote}
Aku terbebaskan dengan baik, terbebaskan dengan sangat baik,\\
 Terbebaskan dari tiga hal yang membuatku menurut:\\
 lumpang, alu,\\
 Dan suamiku yang bungkuk.\\
 Aku terbebas dari kelahiran dan kematian;\\
 Kemelekatan pada kelahiran kembali dihancurkan. 
\end{quote}
\hypertarget{dhammadinnux101}{

\section{1.12 Dhammadinnā}

\label{dhammadinnux101}}
\begin{quote}
Seorang yang bersemangat dan berketetapan\\
 Akan dipenuhi dengan kewaspadaan.\\
 Seorang yang batinnya tidak terikat pada kenikmatan indria\\
 Dikatakan sebagai mengarah menuju arus naik. 
\end{quote}
\hypertarget{visux101khux101}{

\section{1.13 Visākhā}

\label{visux101khux101}}
\begin{quote}
Penuhilah Ajaran Sang Buddha,\\
 Yang setelahnya engkau tidak akan menyesal.\\
 Setelah dengan segera mencuci kakimu,\\
 Duduklah di tempat sunyi untuk bermeditasi. 
\end{quote}
\hypertarget{sumanux101}{

\section{1.14 Sumanā}

\label{sumanux101}}
\begin{quote}
Setelah melihat elemen-elemen sebagai penderitaan\\
 Jangan terlahir kembali lagi\\
 Ketika engkau telah meninggalkan keinginan pada kelahiran kembali,\\
 Engkau akan hidup dengan damai. 
\end{quote}
\hypertarget{uttarux101-1}{

\section{1.15 Uttarā (1)}

\label{uttarux101-1}}
\begin{quote}
Aku terkekang\\
 Dalam jasmani, ucapan, dan pikiran.\\
 Setelah mencabut akar ketagihan dan segalanya,\\
 Aku sejuk dan padam. 
\end{quote}
\hypertarget{sumanux101-yang-meninggalkan-keduniawian-pada-usia-lanjut}{

\section{1.16 Sumanā, Yang Meninggalkan Keduniawian pada Usia Lanjut}

\label{sumanux101-yang-meninggalkan-keduniawian-pada-usia-lanjut}}
\begin{quote}
Tidurlah dengan damai, nyonya tua,\\
 Terbalut jubah yang engkau jahit sendiri;\\
 Karena keinginanmu telah dipadamkan,\\
 Engkau sejuk dan padam. 
\end{quote}
\hypertarget{dhammux101}{

\section{1.17 Dhammā}

\label{dhammux101}}
\begin{quote}
Aku berjalan menerima dana makanan\\
 Walaupun lemah, bersandar pada tongkat.\\
 Kakiku terhuyung-huyung\\
 Dan aku jatuh di tanah di sana.\\
 Melihat bahaya pada tubuh ini,\\
 Batinku terbebaskan. 
\end{quote}
\hypertarget{saux1e41ghux101}{

\section{1.18 Sa\.{m}ghā}

\label{saux1e41ghux101}}
\begin{quote}
Setelah meninggalkan rumahku, anakku, ternakku,\\
 Dan semua yang kucintai, aku meninggalkan keduniawian.\\
 Setelah meninggalkan keinginan dan benci,\\
 Setelah menghalau ketidaktahuan,\\
 Dan setelah mencabut akar ketagihan dan segalanya,\\
 Aku padam dan damai. 
\end{quote}
Kelompok Satu selesai

\hypertarget{abhirux16bpanandux101}{

\chapter{Kelompok Dua}

\section{2.1 Abhirūpanandā}

\label{abhirux16bpanandux101}}
\begin{quote}
Nandā, lihatlah kantong tulang-belulang ini sebagai\\
 Penyakit, kotor, dan busuk.\\
 Dengan pikiran terpusat dan tenang,\\
 Bermeditasilah pada aspek keburukan jasmani.

Bermeditasilah pada tanpa gambaran,\\
 Tinggalkan kecenderungan tersembunyi pada keangkuhan;\\
 Dan ketika engkau memahami keangkuhan,\\
 Engkau akan hidup dengan damai. 
\end{quote}
Demikianlah Sang Buddha secara rutin menasihati bhikkhunī senior Nandā
melalui syair-syair ini.

\hypertarget{jentux101}{

\section{2.2 Jentā}

\label{jentux101}}
\begin{quote}
Di antara tujuh faktor pencerahan,\\
 Jalan untuk mencapai pemadaman,\\
 Aku telah mengembangkannya semua,\\
 Persis seperti yang diajarkan oleh Sang Buddha.

Karena Aku telah melihat Sang Bhagavā,\\
 Dan kantong tulang-belulang ini adalah yang terakhir bagiku.\\
 Transmigrasi melalui kelahiran-kelahiran telah usai,\\
 Sekarang tidak ada lagi kehidupan di masa depan. 
\end{quote}
Demikianlah syair-syair ini dilafalkan oleh bhikkhunī senior Jentā.

\hypertarget{ibunya-sumaux1e45gala}{

\section{2.3 Ibunya Suma\.{n}gala}

\label{ibunya-sumaux1e45gala}}
\begin{quote}
Aku terbebaskan dengan baik, terbebaskan dengan baik,\\
 Terbebaskan dengan sangat baik!\\
 Angin tidak tahu malu dari aluku dulu bertiup;\\
 Periuk kecilku berbau bagai seekor belut.

Sekarang, sehubungan dengan keserakahan dan kebencian:\\
 Aku menghanguskannya hingga mendesis.\\
 Setelah mendatangi bawah pohon, 
\end{quote}
Aku bermeditasi dengan bahagia, dengan berpikir, ``Oh, betapa bahagianya!''

\hypertarget{aux1e0dux1e0dhakux101si}{

\section{2.4 A\d{d}\d{d}hakāsi}

\label{aux1e0dux1e0dhakux101si}}
\begin{quote}
Harga untuk pelayananku\\
 Bernilai negeri Kāsi.\\
 Dengan menetapkan harga itu,\\
 Warga kota membuatku tak ternilai.

Kemudian, dengan menumbuhkan kekecewaan pada bentukku,\\
 Aku menjadi bosan.\\
 Jangan lagi melanjutkan perjalanan,\\
 Bertransmigrasi melalui kelahiran berulang-ulang!\\
 Aku telah merealisasikan tiga pengetahuan,\\
 Dan memenuhi ajaran Sang Buddha. 
\end{quote}
\hypertarget{cittux101}{

\section{2.5 Cittā}

\label{cittux101}}
\begin{quote}
Walaupun aku kurus,\\
 Sakit, dan sangat lemah,\\
 Aku mendaki gunung,\\
 Dengan bersandar pada sebatang tongkat.

Setelah menghamparkan jubah luarku,\\
 Dan membalikkan mangkukku,\\
 Dengan bersandar pada batu,\\
 Aku menghancurkan kumpulan kegelapan. 
\end{quote}
\hypertarget{mettikux101}{

\section{2.6 Mettikā}

\label{mettikux101}}
\begin{quote}
Walaupun dalam kesakitan,\\
 Lemah, kemudaanku telah lama sirna,\\
 Aku mendaki gunung,\\
 Dengan bersandar pada sebatang tongkat.

Setelah menghamparkan jubah luarku\\
 Dan membalikkan mangkukku,\\
 Dengan duduk di atas batu,\\
 Batinku terbebaskan.\\
 Aku telah mencapai tiga pengetahuan,\\
 Dan memenuhi ajaran Sang Buddha. 
\end{quote}
\hypertarget{mittux101-2}{

\section{2.7 Mittā (2)}

\label{mittux101-2}}
\begin{quote}
Aku bergembira dalam kumpulan para dewa,\\
 Setelah melaksanakan uposatha\\
 Yang lengkap dalam seluruh delapan faktor,\\
 Pada tanggal empat belas dan lima belas,

Dan hari ke delapan dwimingguan,\\
 Serta pada hari-hari dwimingguan khusus.\\
 Hari ini aku makan hanya satu kali sehari,\\
 Rambutku tercukur, aku mengenakan jubah luar.\\
 Aku tidak mendambakan kumpulan para dewa,\\
 Karena kesusahan telah lenyap dari batinku. 
\end{quote}
\hypertarget{ibunya-abhaya}{

\newpage

\section{2.8 Ibunya Abhaya}

\label{ibunya-abhaya}}

Putranya, Abhaya Thera:
\begin{quote}
Ibuku sayang, aku memeriksa tubuh ini,\\
 Dari telapak kaki ke atas,\\
 Dan dari ujung rambut ke bawah,\\
 Begitu kotor dan berbau busuk.

Dengan bermeditasi seperti ini,\\
 Segala nafsuku terhapuskan.\\
 Demam nafsu terpotong,\\
 Aku sejuk dan padam. 
\end{quote}
\hypertarget{abhayux101}{

\section{2.9 Abhayā}

\label{abhayux101}}
\begin{quote}
Abhayā, tubuh ini rapuh,\\
 Namun orang-orang biasa melekat padanya.\\
 Aku akan membaringkan tubuh ini,\\
 Dengan sadar dan penuh perhatian.

Walaupun tunduk pada begitu banyak hal menyakitkan,\\
 Aku telah, melalui kecintaanku pada ketekunan,\\
 Mencapai akhir ketagihan,\\
 Dan memenuhi ajaran Sang Buddha. 
\end{quote}
\hypertarget{sux101mux101}{

\newpage

\section{2.10 Sāmā}

\label{sux101mux101}}
\begin{quote}
Empat atau lima kali\\
 Aku meninggalkan tempat kediamanku.\\
 Aku telah gagal menemukan kedamaian batin,\\
 Atau kendali apapun atas pikiranku.\\
 Sekarang ini adalah malam ke delapan\\
 Sejak ketagihan terhapuskan

Walaupun tunduk pada begitu banyak hal menyakitkan,\\
 Aku telah, melalui kecintaanku pada ketekunan,\\
 Mencapai akhir ketagihan,\\
 Dan memenuhi ajaran Sang Buddha. 
\end{quote}
Kelompok Dua selesai

\hypertarget{sux101mux101-yang-lain}{

\chapter{Kelompok Tiga}

\section{3.1 Sāmā yang lain}

\label{sux101mux101-yang-lain}}
\begin{quote}
Dalam dua puluh lima tahun\\
 Sejak aku meninggalkan keduniawian,\\
 Aku tidak ingat bahwa aku pernah menemukan\\
 Ketenangan dalam pikiranku.

Aku telah gagal menemukan kedamaian batin,\\
 Atau kendali apapun atas pikiranku.\\
 Ketika aku mengingat ajaran sang pemenang\\
 Aku didera rasa keterdesakan.

Walaupun tunduk pada begitu banyak hal menyakitkan,\\
 Aku telah, melalui kecintaanku pada ketekunan,\\
 Mencapai akhir ketagihan,\\
 Dan memenuhi ajaran Sang Buddha.\\
 Ini adalah hari ke tujuh\\
 Sejak ketagihanku mengering. 
\end{quote}
\hypertarget{uttamux101}{

\section{3.2 Uttamā}

\label{uttamux101}}
\begin{quote}
Empat atau lima kali\\
 Aku meninggalkan tempat kediamanku.\\
 Aku telah gagal menemukan kedamaian batin,\\
 Atau kendali apapun atas pikiranku.

\newpage

Aku mendatangi seorang bhikkhunī\\
 Yang padanya aku berkeyakinan.\\
 Ia mengajarkan Dhamma kepadaku:\\
 Agregat-agregat, bidang-bidang indria, dan elemen-elemen.

Ketika aku mendengarkan ajarannya,\\
 Sesuai dengan ajarannya,\\
 Aku duduk selama tujuh hari dalam postur yang sama,\\
 Terserap dalam sukacita dan kebahagiaan.\\
 Pada hari ke delapan aku meregangkan kakiku,\\
 Setelah menghancurkan kumpulan kegelapan. 
\end{quote}
\hypertarget{uttamux101-yang-lain}{

\section{3.3 Uttamā yang lain}

\label{uttamux101-yang-lain}}
\begin{quote}
Di antara tujuh faktor pencerahan,\\
 Jalan untuk mencapai pemadaman,\\
 Aku telah mengembangkannya semua,\\
 Persis seperti yang diajarkan oleh Sang Buddha.

Aku mencapai meditasi-meditasi pada kekosongan\\
 Dan tanpa gambaran kapanpun aku inginkan.\\
 Aku adalah putri sah Sang Buddha,\\
 Selalu bersenang dalam padamnya.

Seluruh kenikmatan indria terpotong,\\
 Apakah manusiawi ataupun surgawi.\\
 Transmigrasi melalui kelahiran-kelahiran telah usai,\\
 Sekarang tidak ada lagi kehidupan di masa depan. 
\end{quote}
\hypertarget{dantikux101}{

\section{3.4 Dantikā}

\label{dantikux101}}
\begin{quote}
Meninggalkan meditasi siangku\\
 Di Gunung Puncak Hering,\\
 Aku melihat seekor gajah di tepi sungai\\
 Yang baru saja keluar dari mandinya.

Seorang laki-kaki, dengan membawa tongkat berkait,\\
 Berkata kepada gajah, ``ulurkan kakimu.''\\
 Gajah itu menjulurkan kakinya,\\
 Dan orang itu menungganginya.

Melihat seekor binatang liar begitu jinak,\\
 Tunduk pada kendali manusia,\\
 Pikiranku menjadi tenang:\\
 Itulah sebabnya maka aku pergi ke hutan! 
\end{quote}
\hypertarget{ubbirux12b}{

\section{3.5 Ubbirī}

\label{ubbirux12b}}

Sang Buddha:
\begin{quote}
``Engkau meratap `mohon hiduplah!' di dalam hutan.\\
 Ubbirī, kendalikan dirimu!\\
 Delapan puluh empat ribu manusia,\\
 Semuanya disebut `makhluk hidup',\\
 Telah terbakar di tanah pemakaman ini:\\
 Yang manakah yang engkau ratapi?'' 
\end{quote}
Ubbiri Therī:
\begin{quote}
``Oh! Karena engkau telah mencabut anak panah dariku,\\
 Yang begitu sulit dilihat, tersembunyi dalam batin.\\
 Engkau telah menyingkirkan kesedihan atas putriku\\
 Yang di dalamnya aku pernah terbenam.

Hari ini aku telah mencabut anak panah,\\
 Aku tidak lapar, padam.\\
 Aku berlindung pada Sang Bijaksana itu, Sang Buddha,\\
 Pada ajaranNya, dan pada Sangha.'' 
\end{quote}
\hypertarget{sukkux101}{

\newpage

\section{3.6 Sukkā}

\label{sukkux101}}

Yakkha:
\begin{quote}
``Ada apa dengan orang-orang di Rājagaha?\\
 Mereka terkapar seperti telah meminum fermentasi madu!\\
 Mereka tidak mendatangi Sukkā\\
 Ketika ia mengajarkan ajaran Sang Buddha.

Tetapi para bijaksana---\\
 Mereka seolah-olah meminumnya,\\
 Begitu menarik, lezat dan bergizi,\\
 Bagaikan para pengembara yang menikmati keteduhan awan.''

``Ia dikenal sebagai Sukkā karena kualitas-kualitas cerahnya,\\
 Bebas dari keserakahan, tenang.\\
 Ia membawa jasmani terakhirnya,\\
 Setelah menaklukkan Māra dan tunggangannya.'' 
\end{quote}
\hypertarget{selux101}

\section{3.7 Selā}

\label{selux101}}

Māra:
\begin{quote}
``Tidak ada jalan membebaskan diri dari dunia,\\
 Jadi apa gunanya keterasingan bagimu?\\
 Nikmatilah kesenangan kenikmatan indria;\\
 Jangan menyesalinya kelak.'' 
\end{quote}
Selā Therī:
\begin{quote}
``Kenikmatan indria adalah bagaikan pedang dan tombak\\
 Kelompok-kelompok unsur kehidupan adalah alas pemotongnya.\\
 Apa yang engkau sebut kesenangan indria\\
 Adalah tidak menyenangkan bagiku sekarang.

Kesenangan dihancurkan dalam segala aspek,\\
 Dan kumpulan kegelapan dihancurkan.\\
 Jadi ketahuilah ini, Yang Jahat:\\
 Engkau terkalahkan, pembasmi!'' 
\end{quote}
\hypertarget{somux101}{

\section{3.8 Somā}

\label{somux101}}

Māra:
\begin{quote}
``Keadaan itu sangat menantang;\\
 Ini adalah untuk dicapai oleh para bijaksana.\\
 Tidaklah mungkin bagi seorang perempuan,\\
 Dengan kebijaksanaan dua-jari.'' 
\end{quote}
Somā Therī:
\begin{quote}
``Apa bedanya para perempuan\\
 Jika pikirannya tenang,\\
 Dan pengetahuan muncul\\
 Ketika engkau dengan benar melihat Dhamma.

Kesenangan dihancurkan dalam segala aspek,\\
 Dan kumpulan kegelapan dihancurkan.\\
 Jadi ketahuilah ini, Yang Jahat:\\
 Engkau terkalahkan, pembasmi!'' 
\end{quote}
Kelompok Tiga selesai.

\hypertarget{bhaddux101-kux101pilux101nux12b}{

\chapter{Kelompok Umpat}

\section{4.1 Bhaddā Kāpilānī}

\label{bhaddux101-kux101pilux101nux12b}}
\begin{quote}
Kassapa adalah putra dan pewaris Sang Buddha,\\
 Yang pikirannya tenggelam dalam samādhi.\\
 Ia mengetahui kehidupan-kehidupan lampaunya,\\
 Ia melihat surga dan tempat-tempat sengsara,

Dan telah mencapai akhir kelahiran kembali:\\
 Sang bijaksana itu memiliki pandangan terang sempurna.\\
 Adalah karena ketiga pengetahuan ini\\
 Maka sang brahmana itu adalah seorang penguasa tiga pengetahuan.

Dengan cara yang persis sama, Bhaddā Kāpilānī\\
 Adalah penguasa tiga pengetahuan, penghancur kematian.\\
 Ia membawa jasmani terakhirnya.\\
 Setelah menaklukkan Māra dan tunggangannya.

Melihat bahaya dari dunia ini,\\
 Kami berdua meninggalkan keduniawian.\\
 Sekarang kami jinak, kekotoran kami telah berakhir;\\
 Kami telah menjadi sejuk dan padam. 
\end{quote}
Kelompok Empat selesai

\hypertarget{bhikkhunux12b-tanpa-nama-2}{

\chapter{Kelompok Lima}

\section{5.1 Bhikkhunī Tanpa Nama (2)}

\label{bhikkhunux12b-tanpa-nama-2}}
\begin{quote}
Dalam dua puluh lima tahun\\
 Sejak aku meninggalkan keduniawian\\
 Aku tidak menemukan kedamaian batin,\\
 Bahkan selama sejentikan jari.

Karena gagal menemukan kedamaian batin,\\
 Yang dikotori oleh keinginan indria,\\
 Aku meratap dengan tangan menggapai\\
 Sewaktu memasuki sebuah tempat kediaman.

Aku mendatangi seorang bhikhunī\\
 Yang padanya aku berkeyakinan.\\
 Ia mengajarkan Dhamma kepadaku:\\
 Agregat-agregat, bidang-bidang indria, dan elemen-elemen.

Ketika aku mendengar ajarannya,\\
 Aku mendatangi tempat sunyi.\\
 Aku mengetahui kehidupan-kehidupan lampauku;\\
 Mata-dewaku menjadi murni;

Aku memahami pikiran makhluk-makhluk lain;\\
 Telinga-dewaku menjadi murni;\\
 Aku merealisasikan kekuatan-kekuatan batin,\\
 Dan mencapai akhir kekotoran.\\
 Aku telah merealisasikan enam jenis pengetahuan langsung,\\
 Dan memenuhi ajaran Sang Buddha. 
\end{quote}
\hypertarget{vimalux101-sang-mantan-pelacur}{

\section{5.2 Vimalā, Sang Mantan Pelacur}

\label{vimalux101-sang-mantan-pelacur}}
\begin{quote}
Dimabukkan oleh penampilanku,\\
 Sosokku, kecantikanku, kemasyhuranku,\\
 Dan karena kemudaanku,\\
 Aku merendahkan perempuan lain.

Aku merias tubuhku,\\
 Begitu indah, dirayu oleh orang-orang dungu,\\
 Dan berdiri di pintu rumah bordil,\\
 Bagaikan pemburu memasang perangkap.

Aku bertelanjang untuk mereka,\\
 Memperlihatkan banyak harta karun milikku.\\
 Menciptakan ilusi rumit,\\
 Aku tertawa, menggoda para laki-laki itu.

Hari ini, setelah berkeliling menerima dana makanan,\\
 Dengan kepala tercukur, mengenakan jubah luar,\\
 Aku duduk di bawah sebatang pohon untuk bermeditasi;\\
 Aku telah memperoleh kebebasan dari pemikiran.

Segala belenggu terpotong,\\
 Baik manusiawi maupun surgawi.\\
 Setelah menghapus segala kekotoran,\\
 Aku telah menjadi sejuk dan padam. 
\end{quote}
\hypertarget{sux12bhux101}{

\section{5.3 Sīhā}

\label{sux12bhux101}}
\begin{quote}
Karena perhatian yang tidak benar,\\
 Aku didera oleh keinginan pada kenikmatan indria.\\
 Aku gelisah di masa lalu,\\
 Tidak memiliki kendali atas pikiranku.

Dikuasai oleh kekotoran,\\
 Karena mengejar persepsi keindahan,\\
 Aku tidak memperoleh kedamaian batin.\\
 Di bawah kekuasaan pemikiran-pemikiran bernafsu,

Kurus, pucat, dan lesu,\\
 Selama tujuh tahun aku mengembara,\\
 Penuh kesakitan,\\
 Tidak menemukan kebahagiaan siang atau malam.

Dengan membawa tali\\
 Aku memasuki hutan, dengan berpikir:\\
 ``Lebih baik aku gantung diri\\
 Daripada kembali kepada kehidupan rendah.''

Aku membuat jerat yang kuat\\
 Dan mengikatnya pada dahan pohon.\\
 Dengan mengalungkannya pada leherku,\\
 Batinku terbebaskan. 
\end{quote}
\hypertarget{sundarux12bnandux101}{

\section{5.4 Sundarīnandā}

\label{sundarux12bnandux101}}
\begin{quote}
``Nandā, lihatlah kantong tulang-belulang ini sebagai\\
 Penyakit, kotor, dan busuk.\\
 Dengan pikiran terpusat dan tenang,\\
 Bermeditasilah pada aspek keburukan jasmani:

Sebagamana ini, demikian pula itu,\\
 Sebagaimana itu, demikian pula ini.\\
 Bau busuk menguar darinya,\\
 Ini adalah kesenangan bagi si dungu.''

Dengan memeriksa tubuhku sedemikian,\\
 Tanpa lelah sepanjang siang dan malam,\\
 Setelah mendobraknya\\
 Dengan kebijaksanaanku, aku melihat.

Dengan tekun,\\
 Menyelidiki dengan seksama,\\
 Aku sungguh melihat tubuh ini\\
 Baik di dalam maupun di luar.

Kemudian, dengan menumbuhkan kekecewaan pada tubuhku,\\
 Aku menjadi bosan dalam batin.\\
 Dengan tekun, terlepas,\\
 Aku padam dan damai. 
\end{quote}
\hypertarget{nanduttarux101}{

\section{5.5 Nanduttarā}

\label{nanduttarux101}}
\begin{quote}
Di masa lalu aku menyembah api suci,\\
 Bulan, matahari, dan para dewa.\\
 Setelah mendatangi penyeberangan sungai,\\
 Aku terjun ke dalam air.

Dengan mengambil banyak sumpah,\\
 Aku mencukur setengah kepalaku.\\
 Mempersiapkan tempat tidur di atas tanah,\\
 Aku tidak makan di malam hari.

Aku menyukai hiasan dan riasanku;\\
 Dan dengan mandi dan pijatan dengan minyak,\\
 Aku memuaskan tubuh ini,\\
 Yang didera oleh keinginan pada kenikmatan indria.

Tetapi kemudian aku memperoleh keyakinan,\\
 Dan meninggalkan keduniawian menuju tanpa rumah.\\
 Setelah benar-benar melihat pada tubuh ini,\\
 Keinginan pada kenikmatan indria terhapuskan.

Seluruh kelahiran-kembali terpotong,\\
 Keinginan dan aspirasi juga.\\
 Dengan terlepas dari segala kemelekatan,\\
 Aku mencapai kedamaian batin. 
\end{quote}
\hypertarget{mittux101kux101ux1e37ux12b}{

\section{5.6 Mittākā\d{l}ī}

\label{mittux101kux101ux1e37ux12b}}
\begin{quote}
Setelah meninggalkan keduniawian karena keyakinan\\
 Dari kehidupan awam menuju kehidupan tanpa rumah,\\
 Aku mengembara ke sana-sini,\\
 Iri pada perolehan dan kehormatan.

Dengan mengabaikan tujuan tertinggi,\\
 Aku mengejar yang terendah.\\
 Di bawah kekuasaan kekotoran,\\
 Aku tidak pernah mengetahui tujuan kehidupan pertapaan.

Aku terpukul oleh rasa keterdesakan\\
 Sewaktu sedang duduk di gubukku:\\
 ``Aku berjalan di jalan yang salah,\\
 Di bawah kekuasaan ketagihan.

Hidupku singkat,\\
 Digilas oleh usia tua dan penyakit.\\
 Sebelum tubuh ini hancur,\\
 Tidak ada waktu bagiku untuk lengah.''

Aku memeriksa sesuai dengan kenyataan\\
 Muncul dan lenyapnya agregat-agregat.\\
 Aku berdiri dengan batin terbebaskan,\\
 Setelah memenuhi ajaran Sang Buddha. 
\end{quote}
\hypertarget{sakulux101}{

\section{5.7 Sakulā}

\label{sakulux101}}
\begin{quote}
Sewaktu sedang berdiam di rumah\\
 Aku mendengar ajaran dari seorang bhikkhu.\\
 Aku melihat Dhamma yang tanpa noda,\\
 Pemadaman, keadaan yang tidak dapat musnah.

Dengan meninggalkan putra dan putriku,\\
 Kekayaan dan hasil panen,\\
 Aku memotong rambutku,\\
 Dan meninggalkan keduniawian menuju kehidupan tanpa rumah.

Sebagai seorang calon bhikkhunī,\\
 Aku mengembangkan jalan langsung.\\
 Aku meninggalkan keserakahan dan kebencian,\\
 Bersama dengan kekotoran-kekotoran yang menyertai.

Ketika aku ditahbiskan sepenuhnya menjadi seorang bhikkhunī,\\
 Aku mengingat kehidupan-kehidupan lampauku,\\
 Dan memurnikan mata-dewaku,\\
 Yang tanpa noda dan sepenuhnya terkembang.

Kondisi-kondisi muncul dari penyebab-penyebab, runtuh;\\
 Setelah melihatnya sebagai bukan milikku,\\
 Aku meninggalkan segala kekotoran,\\
 Aku sejuk dan padam. 
\end{quote}
\hypertarget{soux1e47ux101}{

\section{5.8 So\d{n}ā}

\label{soux1e47ux101}}
\begin{quote}
Aku melahirkan sepuluh putra\\
 Dalam bentuk ini, kantong tulang-belulang ini.\\
 Kemudian, ketika lemah dan tua,\\
 Aku mendatangi seorang bhikkhunī.

Ia mengajarkan Dhamma kepadaku:\\
 Agregat-agregat, bidang-bidang indria, dan elemen-elemen.\\
 Ketika aku mendengar ajarannya,\\
 Aku memotong rambutku dan meninggalkan keduniawian.

Ketika aku menjadi seorang calon bhikkhunī,\\
 Mata-dewaku jernih,\\
 Dan aku mengetahui kehidupan-kehidupan lampauku,\\
 Tempat-tempat di mana aku dulu hidup.

Aku bermeditasi pada tanpa-gambaran,\\
 Pikiranku terpusat dan tenang.\\
 Aku mencapai kebebasan segera,\\
 Padam dengan tidak menggenggam.

Agregat-agregat dipahami sepenuhnya;\\
 Agregat-agregat itu ada, tetapi akarnya terpotong.\\
 Terkutuklah engkau, usia tua yang malang!\\
 Sekarang tidak ada lagi kehidupan di masa depan. 
\end{quote}
\hypertarget{bhaddux101-kuux1e47ux1e0dalakesux101}{

\section{5.9 Bhaddā Ku\d{n}\d{d}alakesā}

\label{bhaddux101-kuux1e47ux1e0dalakesux101}}
\begin{quote}
Rambutku terpotong, terbalut lumpur,\\
 Aku biasanya mengembara dengan mengenakan hanya satu jubah.\\
 Aku melihat kesalahan di mana tidak ada kesalahan,\\
 Dan tidak ada kesalahan di mana ada kesalahan.

Meninggalkan meditasi siangku\\
 Di Gunung Puncak Hering,\\
 Aku melihat Sang Buddha yang tanpa noda\\
 Di hadapan Sa\.{n}gha para bhikkhu.

Aku berlutut dan bersujud,\\
 Dan di hadapan Beliau aku merangkapkan tangan.\\
 ``Datanglah Bhaddā,'' Beliau berkata;\\
 Itu adalah penahbisanku.

``Aku mengembara di antara penduduk A\.{n}ga dan Magadha,\\
 Vajjī, Kāsī, dan Kosala.\\
 Aku telah memakan dana makanan dari negeri-negeri itu\\
 Bebas dari utang selama lima puluh tahun.''

``O! ia telah melakukan begitu banyak jasa!\\
 Pengikut awam itu sangat bijaksana.\\
 Ia memberikan jubah kepada Bhaddā,\\
 Yang terbebas dari segala ikatan.'' 
\end{quote}
\hypertarget{paux1e6dux101cux101rux101}{

\section{5.10 Pa\d{t}ācārā}

\label{paux1e6dux101cux101rux101}}
\begin{quote}
Membajak sawah,\\
 Menanam benih di tanah,\\
 Dengan menyokong pasangan dan anak-anak,\\
 Para pemuda memperoleh kekayaan.

Aku sempurna dalam etika,\\
 Dan aku melakukan nasihat Sang Guru,\\
 Dengan tidak malas juga tidak gelisah---\\
 Mengapakah aku tidak mencapai padamnya?

Setelah mencuci kakiku,\\
 Aku mengamati air,\\
 Melihat air pencuci kaki\\
 Mengalir dari tanah yang tinggi ke tanah yang rendah.

Pikiranku menjadi tenang,\\
 Bagaikan kuda berdarah murni yang baik.\\
 Kemudian, sambil membawa pelita,\\
 Aku memasuki kediamanku,\\
 Memeriksa tempat tidurku,\\
 Dan duduk di atas dipanku.

Kemudian, setelah mengambil peniti\\
 Aku mencabut sumbunya.\\
 Kebebasan batinku\\
 Bagaikan padamnya pelita itu. 
\end{quote}
\hypertarget{tiga-puluh-bhikkhunux12b}{

\section{5.11 Tiga Puluh Bhikkhunī}

\label{tiga-puluh-bhikkhunux12b}}
\begin{quote}
``Dengan mengambil alu,\\
 Para pemuda menumbuk jagung.\\
 Dengan menyokong pasangan dan anak-anak,\\
 Para pemuda memperoleh kekayaan.

Lakukanlah nasihat Sang Buddha,\\
 Yang setelahnya engkau tidak akan menyesal.\\
 Setelah cepat-cepat mencuci kakimu,\\
 Duduklah di tempat sunyi untuk bermeditasi.\\
 Tercurah pada ketenangan batin,\\
 lakukanlah nasihat Sang Buddha.''

Setelah mendengar kata-katanya,\\
 Ajaran-ajaran Pa\d{t}ācārā,\\
 Mereka mencuci kaki\\
 Dan mendatangi tempat sunyi\\
 Tercurah pada ketenangan batin.\\
 Mereka melakukan nasihat Sang Buddha.

Pada jaga pertama malam itu,\\
 Mereka mengingat kehidupan-kehidupan lampau mereka.\\
 Pada jaga pertengahan malam itu,\\
 Mereka memurnikan mata-dewa mereka.\\
 Pada jaga terakhir malam itu,\\
 Mereka menghancurkan kumpulan kegelapan.

Mereka bangkit dan bersujud di kakinya:\\
 ``kami telah melakukan nasihatMu;\\
 Kami akan berdiam dengan menghormat padamu,\\
 Bagaikan tiga puluh dewa menghormati Indra,\\
 Tak terkalahkan dalam peperangan.\\
 Para penguasa tiga pengetahuan, kami terbebas dari kekotoran.'' 
\end{quote}
Demikianlah tiga puluh bhikkhunī senior itu menyatakan pencerahan
mereka di hadapan Pa\d{t}ācārā.

\hypertarget{candux101}{

\section{5.12 Candā}

\label{candux101}}
\begin{quote}
Aku bisanya dalam keadaan menyedihkan.\\
 Sebagai seorang janda tanpa anak,\\
 Tanpa teman dan sanak saudara,\\
 Aku tidak memperoleh makanan maupun pakaian.

Aku membawa mangkuk dan tongkat\\
 Dan mengemis dari rumah ke rumah.\\
 Selama tujuh tahun aku mengembara.\\
 Terbakar oleh panas dan dingin.

Kemudian aku melihat seorang bhikkhunī\\
 Menerima makanan dan minuman.\\
 Setelah mendatanginya, aku berkata:\\
 ``Berilah aku pelepasan keduniawan menuju kehidupan tanpa rumah.''

Berkat belas kasihan kepadaku,\\
 Pa\d{t}ācārā memberikan pelepasan keduniawan kepadaku.\\
 Kemudian, setelah menasihatiku,\\
 Ia mendorongku untuk mencapai tujuan tertinggi.

Setelah mendengar kata-katanya,\\
 Aku melakukan nasihatnya.\\
 Nasihat sang nyonya tidak sia-sia:\\
 Sebagai penguasa tiga pengetahuan, aku terbebas dari kekotoran. 
\end{quote}
Kelompok Lima selesai

\hypertarget{paux1e6dux101cux101rux101-yang-memiliki-lima-ratus-pengikut}{

\chapter{Kelompok Enam}

\section{6.1 Pa\d{t}ācārā, Yang Memiliki Lima Ratus Pengikut}

\label{paux1e6dux101cux101rux101-yang-memiliki-lima-ratus-pengikut}}
\begin{quote}
``Ia yang jalannya tidak engkau ketahui,\\
 Tidak dari mana ia datang juga tidak ke mana ia pergi;\\
 Walaupun ia datang entah dari mana,\\
 Engkau menangisi orang itu, meratap, `Oh anakku!'

Tetapi ia yang jalannya engkau ketahui,\\
 Dari mana mereka datang atau ke mana mereka pergi;\\
 Engkau tidak meratapi orang itu---\\
 Demikianlah sifat makhluk-makhluk hidup.

Tanpa diminta ia datang,\\
 Ia pergi tanpa pamit.\\
 Ia pasti telah datang dari suatu tempat,\\
 Dan berdiam selama entah berapa hari.\\
 Ia pergi dari sini melalui satu jalan,\\
 Ia akan pergi dari sana melalui jalan lainnya.

Dengan bepergian dalam bentuk manusia,\\
 Ia akan melanjutkan transmigrasi.\\
 Sebagaimana ia datang, demikian pula ia pergi:\\
 Mengapa meratapi hal itu?''

``Oh! Karena Engkau telah mencabut anak panah dariku,\\
 Yang begitu sulit dilihat, tersembunyi dalam batin.\\
 Engkau telah menyingkirkan kesedihan atas putraku\\
 Yang di dalamnya aku pernah terbenam.

Hari ini aku telah mencabut anak panah,\\
 Aku tidak lapar, padam.\\
 Aku berlindung pada Sang Bijaksana itu, Sang Buddha,\\
 Pada ajaranNya, dan pada Sangha.'' 
\end{quote}
Demikianlah Pa\d{t}ācārā, yang memiliki lima ratus pengikut, menyatakan
pencerahannya.

\hypertarget{vaseux1e6dux1e6dhux12b}{

\section{6.2 Vase\d{t}\d{t}hī}

\label{vaseux1e6dux1e6dhux12b}}
\begin{quote}
Didera kesedihan atas putraku,\\
 Gila, kehilangan akal sehatku,\\
 Telanjang, rambutku beterbangan,\\
 Aku mengembara ke sana-sini.

Aku menetap di tumpukan sampah,\\
 Di pemakaman dan jalan-jalan raya.\\
 Selama tiga tahun aku mengembara,\\
 Dilanda kelaparan dan kehausan.

Kemudian aku bertemu Yang Suci,\\
 Yang telah pergi ke kota Mithilā.\\
 Penjinak mereka yang belum jinak,\\
 Yang Tercerahkan tidak takut pada apapun dari segala penjuru.

Setelah akal sehatku pulih,\\
 Aku bersujud dan duduk.\\
 Berkat belas kasihan\\
 Gotama mengajarkan Dhamma kepadaku.

Setelah mendengarkan ajaran Beliau,\\
 Aku meninggalkan keduniawian menuju kehidupan tanpa rumah.\\
 Mengerahkan diriku pada kata-kata Sang Guru,\\
 Aku merealisasikan keadaan agung.

Segala dukacita terpotong,\\
 Ditinggalkan, berakhir di sini.\\
 Aku telah sepenuhnya memahami landasan\\
 Dari mana kesedihan muncul. 
\end{quote}
\hypertarget{khemux101}{

\section{6.3 Khemā}

\label{khemux101}}

Māra:
\begin{quote}
``Engkau begitu muda dan cantik!\\
 Aku juga muda, seorang pemuda.\\
 Marilah, Khemā, mari kita menikmati\\
 Musik lima-utas.'' 
\end{quote}
Khemā Therī:
\begin{quote}
``Tubuh ini sedang membusuk,\\
 Sakit dan rapuh,\\
 Aku takut dan muak padanya,\\
 Dan aku telah menghapuskan ketagihan indriawi.

Kenikmatan indria adalah bagaikan pedang dan tombak\\
 Agregat-agregat adalah alas pemotongnya.\\
 Apa yang engkau sebut kesenangan indria\\
 Adalah tidak menyenangkan bagiku sekarang.

Kesenangan dihancurkan dalam segala aspek,\\
 Dan kumpulan kegelapan dihancurkan.\\
 Jadi ketahuilah ini, Yang Jahat:\\
 Engkau terkalahkan, pembasmi!''

``Menyembah bintang-bintang,\\
 Melayani api suci di hutan;\\
 Tidak mampu menangkap sifat sejati segala sesuatu,\\
 Bodohnya aku, aku pikir ini adalah kemurnian.

Tetapi sekarang aku menyembah Yang Tercerahkan,\\
 Yang tertinggi di antara manusia.\\
 Dengan melakukan nasihat Sang Guru,\\
 Aku terbebaskan dari segala penderitaan.'' 
\end{quote}
\hypertarget{sujux101tux101}{

\newpage

\section{6.4 Sujātā}

\label{sujux101tux101}}
\begin{quote}
Aku berhiaskan perhiasan dan segala dandanan,\\
 Dengan kalung bunga, dan riasan cendana menumpuk,\\
 Semuanya ditutupi dengan perhiasan\\
 Dan dikelilingi oleh para pelayanku.

Dengan membawa makanan dan minuman,\\
 Yang padat dan lezat dalam jumlah yang tidak sedikit,\\
 Aku meninggalkan rumahku\\
 Dan berangkat menuju taman.

Aku bersenang-senang di sana dan bermain-main,\\
 Dan kemudian, ketika kembali menuju rumahku,\\
 Aku melihat sebuah kediaman monastik,\\
 Maka aku memasuki hutan Añjana di Sāketa.

Melihat Sang Cahaya Dunia,\\
 Aku bersujud dan duduk.\\
 Berkat belas kasihan\\
 Sang Pelihat mengajarkan Dhamma kepadaku.

Ketika aku mendengarkan Sang Petapa Agung,\\
 Aku menembus kebenaran.\\
 Di sana juga aku menemukan Dhamma,\\
 Yang tanpa noda, kondisi tanpa-kematian.

Kemudian, setelah memahami ajaran sejati,\\
 Aku meninggalkan keduniawian menuju kehidupan tanpa rumah.\\
 Aku telah mencapai tiga pengetahuan;\\
 Nasihat Sang Buddha tidaklah sia-sia. 
\end{quote}
\hypertarget{anopamux101}{

\section{6.5 Anopamā}

\label{anopamux101}}
\begin{quote}
Aku dilahirkan dalam keluarga terpandang,\\
 Makmur dan kaya-raya,\\
 Memiliki wajah dan sosok yang cantik;\\
 Putri Majjha yang sejati.

\newpage

Aku diinginkan oleh para pangeran,\\
 Disukai oleh putra-putra dari keluarga kaya.\\
 Seseorang mengutus utusan kepada ayahku:\\
 ``Berikan Anopamā kepadaku!

Berapapun berat badan\\
 putrimu Anopāma,\\
 Aku akan memberimu delapan kali lipat\\
 Dalam bentuk emas dan permata.''

Ketika aku melihat Yang Tercerahkan,\\
 Sesepuh dunia, yang tak terlampaui,\\
 Aku bersujud di kaki Beliau,\\
 Kemudian duduk di satu sisi.

Berkat belas kasihan,\\
 Gotama mengajarkan Dhamma kepadaku.\\
 Selagi duduk di tempat duduk itu,\\
 Aku merealisasikan buah ke tiga.

Kemudian, setelah memotong rambutku,\\
 Aku meninggalkan keduniawian menuju kehidupan tanpa rumah.\\
 Ini adalah hari ke tujuh\\
 Sejak ketagihanku mengering. 
\end{quote}
\hypertarget{mahux101pajux101pati-gotamux12b}{

\section{6.6 Mahāpajāpati Gotamī}

\label{mahux101pajux101pati-gotamux12b}}
\begin{quote}
Oh Buddha, pahlawanku: hormat kepadaMu!\\
 Yang tertinggi di antara semua makhluk,\\
 Yang membebaskan aku dari penderitaan,\\
 Serta banyak makhluk lainnya juga.

Semua penderitaan sepenuhnya dipahami;\\
 Ketagihan---penyebabnya---mengering;\\
 Jalan berunsur delapan telah dikembangkan;\\
 Dan lenyapnya telah direalisasikan olehku.

Sebelumnya aku adalah seorang ibu, seorang putra,\\
 Seorang ayah, seorang saudara, dan seorang nenek.\\
 Tidak mampu menangkap sifat sejati segala ssuatu,\\
 Aku telah bertransmigrasi tanpa hasil.

Sejak aku bertemu Sang Bhagavā,\\
 Kantong tulang-belulang ini adalah yang terakhir bagiku.\\
 Transmigrasi melalui kelahiran-kelahiran telah usai,\\
 Sekarang tidak ada lagi kehidupan di masa depan.

Aku melihat para siswa dalam kerukunan,\\
 Bersemangat dan teguh,\\
 Selalu penuh semangat---\\
 Ini adalah penghormatan kepada para Buddha!

Adalah sungguh demi manfaat banyak makhluk\\
 bahwa Māyā melahirkan Gotama.\\
 Beliau menyingkirkan kumpulan penderitaan\\
 Pada mereka yang didera penyakit dan kematian. 
\end{quote}
\hypertarget{guttux101}{

\section{6.7 Guttā}

\label{guttux101}}
\begin{quote}
Guttā, engkau telah meninggalkan anakmu,\\
 Kekayaanmu, dan semua yang engkau cintai.\\
 Kejarlah tujuan yang karenanya engkau meninggalkan keduniawian;\\
 Jangan jatuh di bawah kendali pikiran.

Makhluk-makhluk tertipu oleh pikiran,\\
 Yang bermain dalam wilayah Māra,\\
 Dungu, mereka melanjutkan perjalanan,\\
 Bertransmigrasi melalui kelahiran kembali yang tak terhitung banyaknya.

Keinginan indria dan niat buruk,\\
 Dan pandangan identitas;\\
 Kekeliruan dalam aturan dan pelaksanaan,\\
 Dan keragu-raguan sebagai yang ke lima.

O bhikkhunī, ketika engkau telah meninggalkan\\
 Belenggu-belenggu yang lebih rendah ini,\\
 Engkau tidak akan kembali\\
 Ke dunia ini lagi.

\newpage

Dan ketika engkau meninggalkan keserakahan,\\
 Keangkuhan, ketidaktahuan, dan kegelisahan,\\
 Setelah memotong belenggu-belenggu,\\
 Engkau akan mengakhiri penderitaan.

Setelah menghapuskan transmigrasi,\\
 Dan sepenuhnya memahami kelahiran kembali,\\
 Tanpa lapar dalam kehidupan ini,\\
 Engkau akan hidup dengan damai. 
\end{quote}
\hypertarget{vijayux101}{

\section{6.8 Vijayā}

\label{vijayux101}}
\begin{quote}
Empat atau lima kali\\
 Aku meninggalkan tempat kediamanku;\\
 Aku telah gagal menemukan kedamaian batin,\\
 Atau kendali apapun atas pikiranku.

Aku mendatangi seorang bhikkhunī\\
 Dan dengan sopan bertanya kepadanya.\\
 Ia mengajarkan Dhamma kepadaku:\\
 Elemen-elemen dan bidang-bidang indria,

Empat Kebenaran Mulia,\\
 Indria-indria dan kekuatan-kekuatan,\\
 Faktor-faktor pencerahan, dan jalan berunsur delapan\\
 Demi mencapai tujuan tertinggi.

Setelah mendengar kata-katanya,\\
 Aku melakukan nasihatnya.\\
 Pada jaga pertama malam itu,\\
 Aku mengingat kehidupan-kehidupan lampauku.

Pada jaga pertengahan malam itu,\\
 Aku memurnikan mata-dewaku.\\
 Pada jaga terakhir malam itu,\\
 Aku menghancurkan kumpulan kegelapan.

Kemudian aku bermeditasi dengan meliputi tubuhku\\
 Dengan sukacita dan kebahagiaan.\\
 Pada hari ke tujuh aku meregangkan kakiku,\\
 Setelah menghancurkan kumpulan kegelapan. 
\end{quote}
Kelompok Enam selesai.

\hypertarget{uttarux101-2}{

\chapter{Kelompok Tuju}

\section{7.1 Uttarā (2)}

\label{uttarux101-2}}

Pa\d{t}ācārā Therī:
\begin{quote}
``Dengan menggunakan alu,\\
 Para pemuda menumbuk jagung.\\
 Dengan menyokong pasangan dan anak-anak,\\
 Para pemuda memperoleh kekayaan.

Bekerja menuruti nasihat Sang Buddha,\\
 Yang setelahnya engkau tidak akan menyesal.\\
 Setelah cepat-cepat mencuci kakimu,\\
 Duduklah di tempat sunyi untuk bermeditasi.

Kokohkan pikiran,\\
 Terpusat dan tenang.\\
 Periksalah kondisi-kondisi\\
 Sebagai bukan milikku, bukan sebagai milikku.'' 
\end{quote}
Uttarā Therī:
\begin{quote}
``Setelah mendengar kata-katanya,\\
 Ajaran-ajaran dari Pa\d{t}ācārā,\\
 Aku mencuci kakiku\\
 Dan pergi ke tempat sunyi.

\newpage

Pada jaga pertama malam itu,\\
 Aku mengingat kehidupan-kehidupan lampauku.\\
 Pada jaga pertengahan malam itu,\\
 Aku memurnikan mata-dewaku.

Pada jaga terakhir malam itu,\\
 Aku menghancurkan kumpulan kegelapan.\\
 Aku bangkit sebagai penguasa tiga pengetahuan:\\
 NasihatMu telah dilakukan.

Aku akan berdiam dengan menghormati engkau\\
 Bagaikan tiga puluh dewa menghormati Sakka,\\
 Yang tak terkalahkan dalam peperangan.\\
 Penguasa tiga pengetahuan, aku bebas dari kekotoran.'' 
\end{quote}
\hypertarget{cux101lux101}{

\section{7.2 Cālā}

\label{cux101lux101}}

Cālā Therī:
\begin{quote}
``Sebagai seorang bhikkhunī dengan indria-indria terkembang,\\
 Setelah menegakkan perhatian,\\
 Aku menembus keadaan damai itu,\\
 Kondisi-kondisi diam yang membahagiakan.'' 
\end{quote}
Māra:
\begin{quote}
``Dalam nama siapakah engkau mencukur kepalamu?\\
 Engkau tampak seperti seorang petapa,\\
 Tetapi engkau tidak meyakini kepercayaan apapun\\
 Mengapakah engkau hidup seolah-olah tersesat?'' 
\end{quote}
Cālā Therī:
\begin{quote}
``Para pengikut kepercayaan lain\\
 Mengandalkan pandangan-pandangan mereka.\\
 Mereka tidak memahami Dhamma\\
 Karena mereka bukan ahli Dhamma.

Tetapi ada seorang yang terlahir dalam suku Sakya,\\
 Sang Buddha yang tanpa tandingan;\\
 Beliau mengajarkan Dhamma kepadaku\\
 Untuk melampaui pandangan-pandangan.

Penderitaan, asal-mula penderitaan,\\
 Melampaui penderitaan,\\
 Dan Jalan Mulia Berunsur Delapan\\
 Yang mengarah menuju diamnya penderitaan.

Setelah mendengar kata-kata Beliau,\\
 Aku dengan bahagia melakukan nasihat Beliau.\\
 Aku telah mencapai tiga pengetahuan\\
 Dan memenuhi ajaran Sang Buddha.

Kesenangan dihancurkan dalam segala aspek,\\
 Dan kumpulan kegelapan dihancurkan.\\
 Jadi ketahuilah ini, Yang Jahat:\\
 Engkau terkalahkan, pembasmi!'' 
\end{quote}
\hypertarget{upacux101lux101}{

\section{7.3 Upacālā}

\label{upacux101lux101}}

Upacālā Therī:
\begin{quote}
``Seorang bhikkhunī dengan indria-indria terkembang,\\
 Penuh perhatian, melihat dengan jelas,\\
 Aku menembus keadaan damai itu,\\
 Yang tidak dilatih oleh para pelanggar.'' 
\end{quote}
Māra:
\begin{quote}
``Mengapa engkau tidak menyetujui kelahiran kembali?\\
 Jika engkau terlahir, maka engkau dapat menikmati kenikmatan indria.\\
 Nikmatilah kebahagiaan kenikmatan indria;\\
 Jangan menyesalinya kelak.'' 
\end{quote}
Upacālā Therī:
\begin{quote}
``Kematian mendatangi mereka yang terlahir;\\
 Dan ketika terlahir mereka jatuh ke dalam penderitaan:\\
 Terpotongnya tangan dan kaki,\\
 Pembunuhan, pengurungan, kesengsaraan.

\newpage

Tetapi ada seorang yang terlahir dalam suku Sakya,\\
 Seorang pemenang yang tercerahkan.\\
 Beliau mengajarkan Dhamma kepadaku\\
 Untuk melampaui kelahiran kembali:

Penderitaan, asal-mula penderitaan,\\
 Melampaui penderitaan,\\
 Dan Jalan Mulia Berunsur Delapan\\
 Yang mengarah menuju diamnya penderitaan.

Setelah mendengar kata-kata Beliau,\\
 Aku dengan bahagia melakukan nasihat Beliau.\\
 Aku telah mencapai tiga pengetahuan\\
 Dan memenuhi ajaran Sang Buddha.

Kesenangan dihancurkan dalam segala aspek,\\
 Dan kumpulan kegelapan dihancurkan.\\
 Jadi ketahuilah ini, Yang Jahat:\\
 Engkau terkalahkan, pembasmi!'' 
\end{quote}
Kelompok Tujuh selesai.

\hypertarget{sux12bsux16bpacux101lux101}{

\chapter{Kelompok Delapan}

\section{8.1 Sīsūpacālā}

\label{sux12bsux16bpacux101lux101}}

Sīsūpacālā Therī:
\begin{quote}
``Seorang bhikkhunī yang sempurna dalam etika,\\
 Organ-organ indrianya terkekang-baik,\\
 Akan merealisasikan keadaan damai,\\
 Yang begitu menarik, lezat dan bergizi.'' 
\end{quote}
Māra:
\begin{quote}
``Ada para Dewa Tiga-puluh-tiga, dan para Dewa Yama;\\
 Juga para dewata bergembira,\\
 Para Dewa yang Suka Mencipta,\\
 Dan para Dewa yang Mengendalikan Ciptaan Dewa lain.\\
 Tetapkan batinmu pada tempat-tempat itu,\\
 Tempat di mana Engkau dulu hidup.'' 
\end{quote}
Sīsūpacālā Therī:
\begin{quote}
``Ada para Dewa Tiga-puluh-tiga, dan para Dewa Yama;\\
 Juga para dewata bergembira,\\
 Para Dewa yang Suka Mencipta,\\
 Dan para Dewa yang Mengendalikan Ciptaan Dewa lain---

Waktu demi waktu, kehidupan demi kehidupan,\\
 Mereka menjadikan identitas sebagai prioritas mereka.\\
 Mereka belum melampaui identitas,\\
 Mereka yang bertransmigrasi melalui kelahiran dan kematian.

Seluruh dunia terbakar,\\
 Seluruh dunia menyala,\\
 Seluruh dunia berkobar,\\
 Seluruh dunia berguncang.

Sang Buddha mengajarkan Dhamma kepadaku,\\
 Yang tak tergoyahkan, tak tertandingi,\\
 Yang jarang didatangi oleh orang-orang biasa;\\
 Batinku menyukai tempat itu.

Setelah mendengar kata-kata Beliau,\\
 Aku dengan bahagia melakukan nasihat Beliau.\\
 Aku telah mencapai tiga pengetahuan,\\
 Dan memenuhi ajaran Sang Buddha.

Kesenangan dihancurkan dalam segala aspek,\\
 Dan kumpulan kegelapan dihancurkan.\\
 Jadi ketahuilah ini, Yang Jahat:\\
 Engkau terkalahkan, pembasmi!'' 
\end{quote}
Kelompok Delapan selesai.

\hypertarget{ibunya-vaux1e0dux1e0dha}{

\newpage
\chapter{Kelompok Sembilan}

\section{9.1 Ibunya Va\d{d}\d{d}ha}

\label{ibunya-vaux1e0dux1e0dha}}

Va\d{d}dhamātu Therī (Ibunya Va\d{d}\d{d}ha Thero):
\begin{quote}
``Va\d{d}\d{d}ha, tolong jangan pernah\\
 Terjebak dalam kekusutan di dunia.\\
 Anakku, jangan lagi mengambil bagian\\
 Dalam penderitaan berulang-ulang.

Karena para bijaksana berdiam dalam bahagia, Va\d{d}dha,\\
 Tidak tergerak, keragu-raguan mereka terpotong,\\
 Sejuk dan jinak,\\
 Dan bebas dari kekotoran.

Va\d{d}\d{d}ha, kembangkanlah sang jalan\\
 Yang telah dilalui oleh para petapa,\\
 Demi pencapaian penglihatan,\\
 Dan demi mengakhiri penderitaan.'' 
\end{quote}
Va\d{d}\d{d}ha Thero (putranya):

``Ibu, engkau berkata dengan begitu yakin\\
 Kepadaku tentang hal ini.\\
 Ibuku sayang, aku tidak bisa tidak berpikir\\
 Bahwa tidak ada kekusutan padamu.''

\newpage

Va\d{d}dhamātu Therī (Ibunya Va\d{d}\d{d}ha Thero):
\begin{quote}
``Va\d{d}dha, tidak ada sedikitpun\\
 Kekusutan terdapat padaku\\
 Untuk kondisi apapun juga\\
 Apakah rendah, tinggi, atau menengah.

Segala kekotoran telah berakhir padaku,\\
 Dengan bermeditasi dan tekun.\\
 Aku telah mencapai tiga pengetahuan\\
 Dan memenuhi ajaran Sang Buddha.'' 
\end{quote}
Va\d{d}\d{d}ha Thero (putranya):
\begin{quote}
``Sungguh baik tongkat kendali\\
 Yang digunakan ibuku untuk mendorongku!\\
 Berkat belas kasihnya, ia mengucapkan\\
 Syair-syair tentang tujuan tertinggi.

Setelah mendengar kata-katanya.\\
 Karena diajari oleh ibuku,\\
 Aku terpukul oleh keterdesakan yang benar\\
 Untuk mencari suaka

Dengan berjuang, teguh,\\
 Tanpa lelah sepanjang siang dan malam,\\
 Dipacu oleh ibuku,\\
 Aku merealisasikan kedamaian tertinggi.'' 
\end{quote}
Kelompok Sembilan selesai

\hypertarget{kisux101gotami}{

\newpage
\chapter{Kelompok Sepuluh}

\section{10.1 Kisāgotami}

\label{kisux101gotami}}
\begin{quote}
``Dengan menunjukkan bagaimana dunia ini bekerja,\\
 Para bijaksana memuji pertemanan yang baik.\\
 Dengan bergaul dengan teman-teman baik\\
 Bahkan seorang dungu dapat menjadi cerdik.

Bergaul dengan orang-orang baik,\\
 Karena itu adalah bagaimana kebijaksanaan tumbuh.\\
 Jika engkau bergaul dengan orang-orang baik,\\
 Maka engkau akan terebebas dari segala penderitaan.

Dan engkau akan memahami penderitaan,\\
 Asal-mula dan lenyapnya,\\
 Sang Jalan berunsur delapan,\\
 Dan juga Empat Kebenaran Mulia.''

``\,`Kehidupan seorang perempuan adalah menyakitkan,'\\
 Sang Buddha menjelaskan, tuntunan bagi mereka yang ingin berlatih,\\
 `dan khususnya bagi seorang istri yang berbagi suami.\\
 Setelah melahirkan hanya satu kali,

Beberapa perempuan bahkan memotong lehernya sendiri,\\
 Sementara perempuan-perempuan yang lebih halus meminum racun.\\
 Karena bersalah membunuh orang,\\
 Mereka mengalami kehancuran baik di sini maupun di alam lain.'\,''

\newpage

``Aku sedang dalam perjalanan dan menjelang bersalin,\\
 Ketika aku melihat suamiku mati.\\
 Aku melahirkan di sana di jalan\\
 Sebelum sampai di rumahku.

Kedua anakku mati,\\
 Dan di jalan suamiku terbaring mati -- oh malangnya aku!\\
 Ibu, ayah, dan kakak laki-laki\\
 Semuanya terbakar dalam tumpukan kayu yang sama.''

``Oh malangnya engkau yang kehilangan keluarga,\\
 Penderitaanmu tidak terukur;\\
 Engkau telah meneteskan air mata\\
 Selama ribuan kehidupan.''

``Selagi berada di tanah pemakaman.\\
 Aku melihat daging anakku dimakan.\\
 Dengan keluarga hancur, dikutuk oleh orang-orang lain,\\
 Dan suamiku mati, aku merealisasikan tanpa-kematian.

Aku telah mengembangkan Jalan Mulia Berunsur Delapan\\
 Yang mengarah menuju tanpa-kematian.\\
 Aku telah merealisasikan padamnya,\\
 Seperti yang terlihat dalam cermin Dhamma.

Aku telah mencabut anak panah,\\
 Menurunkan beban, dan melakukan apa yang perlu dilakukan.''\\
 Bhikkhunī senior Kisāgotamī,\\
 Yang batinnya terbebaskan, mengatakan ini. 
\end{quote}
Kelompok Sebelas selesai

\hypertarget{uppalavaux1e47ux1e47ux101}{

\newpage
\chapter{Kelompok Sebelas}

\section{11.1 Uppalava\d{n}\d{n}ā}

\label{uppalavaux1e47ux1e47ux101}}

Uppalava\d{n}\d{n}ā Therī (Komentar menyatakan bahwa tiga syair pertama
diucapkan oleh ibunya):
\begin{quote}
``Kami berdua adalah istri-istri dari suami yang sama,\\
 Walaupun kami adalah ibu dan anak.\\
 Aku terpukul oleh rasa keterdesakan,\\
 Yang begitu mengagumkan dan membuat bulu-badan berdiri!

Terkutuklah kenikmatan-kenikmatan indria yang kotor itu,\\
 Yang begitu menjijikkan dan berduri,\\
 Di mana kami, ibu dan anak,\\
 Harus menjadi istri-istri dari suami yang sama.

Melihat bahaya dalam kenikmatan-kenikmatan indria,\\
 Melihat pelepasan keduniawian sebagai suaka,\\
 Aku meninggalkan keduniawian di Rājagaha\\
 Dari kehidupan awam menuju kehidupan tanpa rumah.

Aku mengetahui kehidupan-kehidupan lampauku;\\
 Mata-dewaku murni;\\
 Aku memahami pikiran makhluk-makhluk lain;\\
 Telinga-dewaku murni;

\newpage

Aku merealisasikan kekuatan-kekuatan batin,\\
 Dan mencapai akhir kekotoran\\
 Aku telah merealisasikan enam jenis pengetahuan langsung,\\
 Dan memenuhi ajaran Sang Buddha.

Aku menciptakan kereta dengan empat kuda\\
 Melalui kekuatan batinku.\\
 Kemudian aku bersujud di kaki Sang Buddha,\\
 Sang pelindung dunia yang mulia.'' 
\end{quote}
Māra:
\begin{quote}
``Engkau datang ke pohon sal yang bermahkotakan bunga-bunga,\\
 Dan berdiri sendirian di akarnya\\
 Tetapi engkau tidak memiliki pendamping bersamamu,\\
 Gadis bodoh, tidakkah engkau takut pada penjahat?'' 
\end{quote}
Uppalava\d{n}\d{n}ā Therī:
\begin{quote}
``Bahkan jika 100.000 penjahat seperti ini\\
 Berkumpul,\\
 Aku tidak akan tergerak sehelai rambut pun juga tidak gemetar,\\
 Apa yang dapat engkau lakukan sendirian kepadaku, Māra?

Aku akan lenyap,\\
 Atau aku akan memasuki perutmu;\\
 Aku dapat berdiri di antara kedua alis matamu\\
 Dan engkau tidak akan melihatku.

Aku adalah penguasa bagi pikiranku,\\
 Aku telah dengan baik mengembangkan landasan-landasan kekuatan batin.\\
 Aku telah merealisasikan enam jenis pengetahuan langsung,\\
 Dan memenuhi ajaran Sang Buddha.

Kenikmatan indria adalah bagaikan pedang dan tombak\\
 Agregat-agregat adalah alas pemotongnya.\\
 Apa yang engkau sebut kesenangan indria\\
 Adalah tidak menyenangkan bagiku sekarang.

\newpage

Kesenangan dihancurkan dalam segala aspek,\\
 Dan kumpulan kegelapan dihancurkan.\\
 Jadi ketahuilah ini, Yang Jahat:\\
 Engkau terkalahkan, pembasmi!'' 
\end{quote}
Kelompok Dua Belas Selesai.

\hypertarget{puux1e47ux1e47ikux101}{

\newpage
\chapter{Kelompok Enam Belas}

\section{12.1 Pu\d{n}\d{n}ikā}

\label{puux1e47ux1e47ikux101}}

Pu\d{n}\d{n}ā Therī:
\begin{quote}
``Aku adalah seorang pembawa-air. Bahkan ketika cuaca dingin,\\
 Aku harus selalu terjun ke dalam air;\\
 Aku takut akan dipukul oleh para nyonya khattiya,\\
 Terganggu oleh ketakutan pada pukulan dan kemarahan.

Brahmana, apakah yang engkau takutkan,\\
 Sehingga engkau selalu terjun ke dalam air,\\
 Tangan dan kakimu gemetar\\
 Dalam dingin yang membeku?'' 
\end{quote}
Brāhma\d{n}a:
\begin{quote}
``Oh, tetapi engkau sudah tahu,\\
 Nyonya Pu\d{n}\d{n}ikā, ketika engkau bertanya kepadaku:\\
 Aku sedang melakukan perbuatan-perbuatan baik,\\
 Untuk menghalangi kejahatan yang telah kulakukan.

Siapapun juga tua atau muda\\
 yang melakukan perbuatan jahat,\\
 dengan pencucian dalam air mereka akan\\
 terbebas dari perbuatan jahat mereka.'' 

\newpage
\end{quote}
Pu\d{n}\d{n}ā Therī:
\begin{quote}
``Siapakah di dunia ini yang memberitahukan ini kepadamu,\\
 Seorang dungu kepada orang dungu lainnya:\\
 `bahwa, dengan pencucian dalam air seseorang\\
 terbebas dari perbuatan jahat.'

Kalau begitu mengapa mereka semua tidak pergi ke surga:\\
 Semua kodok dan kura-kura,\\
 Ular, buaya,\\
 Serta penghuni air lainnya juga?

Penjagal kambing dan babi,\\
 Nelayan, penjerat binatang,\\
 Penjahat, algojo,\\
 Dan para pelaku kejahatan lainnya:\\
 Dengan pencucian dalam air mereka juga akan\\
 Terbebas dari perbuatan jahat mereka.

Jika sungai-sungai ini mencuci\\
 Perbuatan-perbuatan buruk masa lalu,\\
 Maka sungai itu juga mencuci kebaikan,\\
 Dan karena itu engkau akan menjadi tidak termasuk.

Brahmana, hal yang engkau takutkan,\\
 Ketika engkau terjun ke dalam air,\\
 Jangan lakukan hal itu,\\
 Jangan sampai dingin melukai kulitmu.'' 
\end{quote}
Brāhma\d{n}a:
\begin{quote}
``Aku telah berada di jalan yang salah,\\
 Dan engkau telah membimbingku pada jalan mulia.\\
 Nyonya, aku mempersembahkan kepadamu\\
 Kain pencucian ini.'' 
\end{quote}
Pu\d{n}\d{n}ā Therī:
\begin{quote}
``Simpanlah kain itu untukmu sendiri,\\
 Aku tidak menginginkannya.\\
 Jika engkau takut pada penderitaan,\\
 Jika engkau tidak menyukai penderitaan.

Maka jangan lakukan perbuatan buruk\\
 Apakah secara terbuka ataupun secara diam-diam.\\
 Jika engkau harus melakukan perbuatan buruk,\\
 Atau engkau sedang melakukannya sekarang,

Engkau tidak akan terbebas dari penderitaan,\\
 Walaupun engkau terbang dan melarikan diri.\\
 Jika engkau takut pada penderitaan,\\
 Jika engkau tidak menyukai penderitaan.

Pergilah berlindung kepada Sang Buddha, yang seimbang,\\
 Kepada ajaran Beliau dan kepada Sangha.\\
 Terimalah aturan-aturan latihan,\\
 Itu baik untukmu.'' 
\end{quote}
Brāhma\d{n}a:
\begin{quote}
``Aku pergi berlindung kepada Sang Buddha, yang seimbang,\\
 Kepada ajaran Beliau dan kepada Sangha.\\
 Aku menerima aturan-aturan latihan,\\
 Itu baik untukku.

Di masa lalu aku adalah kerabat Brahmā,\\
 Hari ini aku benar-benar adalah seorang brahmana!\\
 Aku adalah penguasa tiga pengetahuan, sempurna dalam kebijaksanaan,\\
 Aku terpelajar dan seorang yang telah tercuci.'' 
\end{quote}
Kelompok Enam Belas selesai.

\hypertarget{ambapux101lux12b}{

\newpage
\chapter{Kelompok Dua Puluh}

\section{13.1 Ambapālī}

\label{ambapux101lux12b}}
\begin{quote}
Rambutku hitam bagaikan kumbang,\\
 Anggun dengan ujung yang ikal;\\
 Sekarang setelah tua, jadi seperti kulit rami---\\
 Kata-kata dari sang pembabar kebenaran telah terkonfirmasi.

Bermahkotakan bunga-bunga,\\
 Kepalaku harum bagaikan peti wewangian;\\
 Sekarang setelah tua, jadi berbau bagaikan bulu anjing---\\
 Kata-kata dari sang pembabar kebenaran telah terkonfirmasi.

Rambutku dulu tebal bagaikan hutan yang ditanam dengan baik,\\
 Bersinar, dipisahkan oleh sisir dan penjepit rambut;\\
 Sekarang setelah tua, menjadi jarang dan rontok---\\
 Kata-kata dari sang pembabar kebenaran telah terkonfirmasi.

Dengan kepangan hitam dan pita emas,\\
 Rambut itu dulu begitu indah, dihias dengan jalinan;\\
 Sekarang setelah tua, kepalaku membotak---\\
 Kata-kata dari sang pembabar kebenaran telah terkonfirmasi.

Alis mataku dulu tampak cantik,\\
 Bagaikan bulan sabit yang dilukis oleh seorang pelukis;\\
 Sekarang setelah tua, alis itu melengkung karena keriput---\\
 Kata-kata dari sang pembabar kebenaran telah terkonfirmasi.

\newpage

Mataku dulu bersinar bagai permata,\\
 Besar dan biru tua;\\
 Dirusak oleh penuaan, mata itu tidak lagi bersinar---\\
 Kata-kata dari sang pembabar kebenaran telah terkonfirmasi.

Hidungku dulu seperti puncak yang sempurna,\\
 Indah pada puncak kemudaanku;\\
 Sekarang setelah tua, layu bagaikan cabai;\\
 Kata-kata dari sang pembabar kebenaran r telah terkonfirmasi.

Daun telingaku dulu sangat cantik,\\
 Bagaikan gelang yang dibentuk dengan indah;\\
 Sekarang setelah tua, menjadi melengkung karena keriput---\\
 Kata-kata dari sang pembabar kebenaran telah terkonfirmasi.

Gigiku dulu begitu indah,\\
 Cerah bagaikan bunga melati;\\
 Sekarang setelah tua, menjadi rusak dan menguning---\\
 Kata-kata dari sang pembabar kebenaran telah terkonfirmasi.

Nyanyianku dulu merdu bagaikan kicauan burung\\
 Yang mengembara di dalam hutan;\\
 Sekarang setelah tua, menjadi gagap dan parau---\\
 Kata-kata dari sang pembabar kebenaran telah terkonfirmasi.

Leherku dulu sangat indah,\\
 Bagaikan kulit kerang yang digosok;\\
 Sekarang setelah tua, menjadi bungkuk---\\
 Kata-kata dari sang pembabar kebenaran telah terkonfirmasi.

Lenganku dulu sangat indah,\\
 Bagaikan palang bundar;\\
 Sekarang setelah tua, lengan itu melengkung bagaikan pohon bunga
trompet---\\
 Kata-kata dari sang pembabar kebenaran telah terkonfirmasi.

Tanganku dulu sangat indah,\\
 Berhiaskan cincin emas yang indah;\\
 Sekarang setelah tua, menjadi merah bagaikan buah bit---\\
 Kata-kata dari sang pembabar kebenaran telah terkonfirmasi.

Payudaraku dulu sangat indah,\\
 Menggembung, bulat, penuh, dan mendongak;\\
 Sekarang terkulai bagaikan kantong air---\\
 Kata-kata dari sang pembabar kebenaran telah terkonfirmasi.

Tubuhku dulu sangat indah,\\
 Bagaikan lempeng emas yang digosok;\\
 Sekarang tertutup oleh keriput halus---\\
 Kata-kata dari sang pembabar kebenaran telah terkonfirmasi.

Kedua pahaku dulu sangat indah,\\
 Bagaikan belalai gajah;\\
 Sekarang setelah tua, menjadi seperti bambu---\\
 Kata-kata dari sang pembabar kebenaran telah terkonfirmasi.

Betisku dulu sangat indah,\\
 Berhiaskan gelang-gelang kaki yang manis;\\
 Sekarang setelah tua, menjadi seperti batang wijen---\\
 Kata-kata dari sang pembabar kebenaran telah terkonfirmasi.

Kedua kakiku dulu sangat indah,\\
 Montok bagai diisi kapas;\\
 Sekarang setelah tua, kaki itu pecah dan keriput---\\
 Kata-kata dari sang pembabar kebenaran telah terkonfirmasi.

Kantong tulang-belulang ini pernah demikian,\\
 Tetapi sekarang telah layu, rumah bagi begitu banyak penyakit;\\
 Bagaikan rumah tua dengan plaster rontok---\\
 Kata-kata dari sang pembabar kebenaran telah terkonfirmasi. 
\end{quote}
\hypertarget{rohinux12b}{

\section{13.2 Rohinī}

\label{rohinux12b}}

Ayahnya Rohinī Therī:
\begin{quote}
``Engkau jatuh terlelap dengan mengatakan `petapa';\\
 Engkau terjaga dengan mengatakan `petapa';\\
 Engkau hanya memuji para petapa, nyonya---\\
 Pasti engkau akan menjadi seorang petapa.

Engkau memberikan kepada para petapa\\
 Makanan dan minuman berlimpah.\\
 Aku bertanya kepadamu sekarang, Rohinī:\\
 Mengapa engkau meyukai petapa?

Mereka tidak suka bekerja, mereka malas,\\
 Mereka hidup dari derma;\\
 Selalu mencari, serakah pada manisan---\\
 Jadi mengapa engkau menyukai petapa?'' 
\end{quote}
Rohinī Therī:
\begin{quote}
``Ayah, sudah sejak lama\\
 Engkau menanyaiku tentang para petapa\\
 Aku akan memuji mereka untukmu\\
 Kebijaksanaan, etika, dan semangat mereka.

Mereka suka bekerja, mereka tidak malas;\\
 Dengan meninggalkan keserakahan dan kebencian,\\
 mereka melakukan jenis pekerjaan terbaik---\\
 itulah sebabnya mengapa aku menyukai para petapa.

Sehubungan dengan tiga akar kejahatan,\\
 Melalui perbuatan-perbuatan murni mereka meruntuhkannya.\\
 Mereka telah meninggalkan segala kejahatan---\\
 Itulah sebabnya mengapa aku menyukai para petapa.

Perbuatan jasmani mereka murni;\\
 Perbuatan ucapan mereka juga demikian;\\
 Perbuatan pikiran mereka murni---\\
 Itulah sebabnya mengapa aku menyukai para petapa.

Sempurna bagaikan kulit kerang,\\
 Mereka murni luar dalam,\\
 Penuh dengan kualitas-kualitas cerah---\\
 Itulah sebabnya mengapa aku menyukai para petapa.

Mereka terpelajar dan menghafalkan ajaran,\\
 Mulia, hidup selayaknya,\\
 Mengajarkan teks dan maknanya:\\
 Itulah sebabnya mengapa aku menyukai para petapa.

Mereka terpelajar dan menghafalkan ajaran,\\
 Mulia, hidup selayaknya,\\
 Dengan pikiran terpusat, dan penuh perhatian---\\
 Itulah sebabnya mengapa aku menyukai para petapa.

Melakukan perjalanan jauh, dan penuh perhatian,\\
 Bijaksana dalam memberi nasihat, dan stabil,\\
 Mereka memahami akhir penderitaan---\\
 Itulah sebabnya mengapa aku menyukai para petapa.

Ketika mereka meninggalkan sebuah desa,\\
 Mereka tidak melihat ke belakang dengan kerinduan,\\
 Melainkan berjalan maju tanpa peduli---\\
 Itulah sebabnya mengapa aku menyukai para petapa.

Mereka tidak menimbun benda-benda di dalam gudang,\\
 Juga tidak di dalam kendi-kendi atau keranjang-keranjang.\\
 Mereka mencari makanan yang dipersiapkan oleh orang lain---\\
 Itulah sebabnya mengapa aku menyukai para petapa.

Mereka tidak menerima perak,\\
 Atau emas apakah dalam bentuk uang atau bukan-uang;\\
 Makan dari apapun yang diterima pada hari itu,\\
 Itulah sebabnya mengapa aku menyukai para petapa.

Mereka telah meninggalkan keduniawian dari keluarga yang berbeda-beda,\\
 Bahkan dari negeri-negeri berbeda,\\
 Namun mereka saling mencintai satu sama lain---\\
 Itulah sebabnya mengapa aku menyukai para petapa.'' 
\end{quote}
Ayahnya Rohinī Therī:
\begin{quote}
``Rohinī sayang, sungguh adalah demi manfaat bagi kami\\
 Maka engkau terlahir dalam keluarga kami!\\
 Engkau memiliki keyakinan dan penghormatan tinggi\\
 Kepada Sang Buddha, ajaran Beliau, dan Sangha.

Karena engkau memahami\\
 Lahan jasa yang tertinggi ini\\
 Mulai saat ini para petapa itu juga akan\\
 Menerima persembahan religius dari kami.'' 
\end{quote}
Rohinī Therī:
\begin{quote}
``Di sanalah kalian harus menempatkan pengorbanan kalian,\\
 Dan itu akan berlimpah.\\
 Jika engkau takut pada penderitaan,\\
 Jika engkau tidak menyukai penderitaan,

Pergilah berlindung kepada Sang Buddha, yang seimbang,\\
 Kepada ajaran Beliau dan kepada Sangha.\\
 Terimalah aturan-aturan latihan,\\
 Itu baik untukmu.'' 
\end{quote}
Ayahnya Rohinī Therī:
\begin{quote}
``Aku pergi berlindung kepada Sang Buddha, yang seimbang,\\
 Kepada ajaran Beliau dan kepada Sangha.\\
 Aku menerima aturan-aturan latihan,\\
 Itu baik untukku.

Di masa lalu aku adalah kerabat Brahmā,\\
 Sekarang aku benar-benar adalah seorang brahmana!\\
 Aku adalah penguasa tiga pengetahuan, seorang terpelajar sejati,\\
 Aku adalah penguasa-pengetahuan, seorang yang tercuci.'' 
\end{quote}
\hypertarget{cux101pux101}{

\section{13.3 Cāpā}

\label{cux101pux101}}

Ājīvaka Upaka Thero (suami dari Cāpā Therī):
\begin{quote}
``Dulu aku membawa tongkat petapa,\\
 Tetapi sekarang aku berburu rusa.\\
 Keinginanku membuatku tidak mampu menyeberang\\
 Dari rawa-rawa mengerikan menuju pantai seberang.

Karena berpikir aku begitu mencintainya,\\
 Cāpā membahagiakan putra kami.\\
 Setelah memotong belenggu Cāpā,\\
 Sekali lagi aku akan meninggalkan keduniawian.'' 
\end{quote}
Cāpā Therī:
\begin{quote}
``Jangan marah padaku, pahlawan besar!\\
 Jangan marah padaku, sang bijaksana agung!\\
 Jika engkau terbenam dalam kemarahan maka engkau tidak dapat mempertahankan
kemurnian,\\
 Apalagi berlatih keras.'' 

\newpage
\end{quote}
Ājīvaka Upaka Thero (suami dari Cāpā Therī):
\begin{quote}
``Aku akan meninggalkan Nālā!\\
 Karena siapakah yang mau menetap di sini di Nālā!\\
 Dengan sosok mereka, perangkap perempuan\\
 Para petapa yang hidup dengan benar.'' 
\end{quote}
Cāpā Therī:
\begin{quote}
``Mohon, Kā\d{l}a, kembalilah kepadaku.\\
 Nikmatilah kenikmatan seperti sebelumnya.\\
 Aku akan berada di bawah kendalimu,\\
 Bersama dengan sanak saudara yang kumiliki.'' 
\end{quote}
Ājīvaka Upaka Thero (suami dari Cāpā Therī):
\begin{quote}
``Cāpā, bahkan jika hanya seperempat\\
 Dari apa yang engkau katakan adalah benar,\\
 Itu adalah hal yang sangat baik\\
 Bagi seorang laki-laki yang mencintaimu!'' 
\end{quote}
Cāpā Therī:
\begin{quote}
``Kā\d{l}a, aku seperti bunga iris yang bertunas\\
 Yang berbunga di puncak gunung,\\
 Bagaikan buah delima yang mekar,\\
 Bagaikan pohon bunga trompet di sebuah pulau;

Tangan dan kakiku diurapi dengan cendana kuning,\\
 Dan aku mengenakan kain Kāsi terbaik:\\
 Sementara aku begitu cantik,\\
 Bagaimana engkau dapat meninggalkan aku dan pergi?'' 
\end{quote}
Ājīvaka Upaka Thero (suami dari Cāpā Therī):
\begin{quote}
``Engkau bagaikan seorang pemburu burung\\
 Yang ingin menangkap burung;\\
 Tetapi engkau tidak akan menjebakku\\
 Dengan bentukmu yang memikat.'' 

\newpage
\end{quote}
Cāpā Therī:
\begin{quote}
``Tetapi anak ini, buahku,\\
 Dilahirkan oleh engkau, Kā\d{l}a.\\
 Ketika aku memiliki anak ini,\\
 Bagaimana engkau dapat meninggalkan aku dan pergi?'' 
\end{quote}
Ājīvaka Upaka Thero (suami dari Cāpā Therī):
\begin{quote}
``Para bijaksana meninggalkan\\
 Anak-anak, keluarga, dan kekayaan.\\
 Para pahlawan besar meninggalkan keduniawian\\
 Bagaikan gajah-gajah mematahkan belenggunya.'' 
\end{quote}
Cāpā Therī:
\begin{quote}
``Sekarang, putramu ini:\\
 Aku akan memukulnya di tanah di sini,\\
 Dengan tongkat atau pisau!\\
 Karena bersedih atas putramu, engkau tidak akan pergi.'' 
\end{quote}
Ājīvaka Upaka Thero (suami dari Cāpā Therī):
\begin{quote}
``Bahkan jika engkau memberikan putra kita untuk dimakan\\
 Oleh serigala dan anjing,\\
 Aku tidak akan pernah kembali lagi, engkau perempuan jalang,\\
 Bahkan tidak demi anak ini.'' 
\end{quote}
Cāpā Therī:
\begin{quote}
``Baiklah, tuan, katakan kepadaku,\\
 Ke manakah engkau akan pergi, Kā\d{l}a?\\
 Ke desa atau pemukiman,\\
 Kota atau ibukota apakah?'' 
\end{quote}
Ājīvaka Upaka Thero (suami dari Cāpā Therī):
\begin{quote}
``Dulu kami memiliki pengikut,\\
 Kami bukan petapa, kami hanya berpikir bahwa kami adalah petapa.\\
 Kami mengembara dari desa ke desa,\\
 Ke kota-kota dan ibukota.

Tetapi sekarang Sang Bhagavā, Sang Buddha,\\
 Di tepi Sungai Nerañjara,\\
 Mengajarkan Dhamma sehingga makhluk-makhluk hidup\\
 Dapat meninggalkan segala penderitaan.\\
 Aku akan pergi ke hadapan Beliau,\\
 Beliau akan menjadi guruku.'' 
\end{quote}
Cāpā Therī:
\begin{quote}
``Sekarang mohon sampaikan hormatku\\
 Kepada pelindung dunia yang tertinggi.\\
 Kelilingi Beliau pada sisi kananmu,\\
 Persembahkan donasi religiusku.'' 
\end{quote}
Ājīvaka Upaka Thero (suami dari Cāpā Therī):
\begin{quote}
``Ini adalah hal yang benar untuk dilakukan,\\
 Seperti apa yang engkau katakan kepadaku.\\
 Aku akan menyampaikan hormatmu\\
 Kepada pelindung dunia yang tertinggi.\\
 Dengan mengelilingnya pada sisi kananku,\\
 Aku akan mempersembahkan donasi religiusmu.'' 
\end{quote}
Penyusun:
\begin{quote}
Kemudian Kā\d{l}a pergi\\
 Ke tepi Sungai Nerañjara\\
 Ia melihat Yang Tercerahkan\\
 Sedang mengajarkan keadaan tanpa-kematian:

Penderitaan, asal-mula penderitaan,\\
 Melampaui penderitaan,\\
 Dan Jalan Mulia Berunsur Delapan\\
 Yang mengarah menuju diamnya penderitaan.

Ia bersujud di kaki Beliau,\\
 Mengelilingi Beliau pada sisi kanannya,\\
 Dan menyampaikan persembahan Cāpā;\\
 Kemudian ia meninggalkan keduniawian menuju kehidupan tanpa rumah.\\
 Ia mencapai tiga pengetahuan,\\
 Dan memenuhi ajaran Sang Buddha. 
\end{quote}
\hypertarget{sundarux12b}{

\section{13.4 Sundarī}

\label{sundarux12b}}

Brāhma\d{n}a Sujāta (ayahnya Sundarī Therī) bertanya kepada Vāse\d{t}\d{t}hi
Therī:
\begin{quote}
``Sebelumnya, ketika anak-anakmu meninggal dunia,\\
 Engkau akan membiarkan mereka untuk dimakan.\\
 Sepanjang siang dan malam\\
 Engkau akan dilanda keputus-asaan

Hari ini, nyonya brahmana, engkau telah membiarkan\\
 Tujuh anakmu semuanya untuk dimakan;\\
 Vāse\d{t}\d{t}hī, apakah alasannya mengapa\\
 Engkau tidak dilanda keputus-asaan?'' 
\end{quote}
Vase\d{t}\d{t}hī Therī:
\begin{quote}
``Ratusan putraku,\\
 Ratusan lingkaran keluarga,\\
 Dari aku dan engkau, brahmana,\\
 Telah dimakan di masa lalu.

Setelah mengetahui jalan membebaskan diri\\
 Dari kelahiran kembali dan kematian\\
 Aku tidak bersedih atau meratap,\\
 Juga aku tidak putus asa.'' 
\end{quote}
Brāhma\d{n}a Sujāta (ayahnya Sundarī Therī) bertanya kepada Vāse\d{t}\d{t}hi
Therī:
\begin{quote}
``Wow, Vase\d{t}\d{t}hī, kata-kata yang engkau ucapkan\\
 Sungguh mengagumkan!\\
 Ajaran siapakah yang engkau pahami\\
 Sehingga engkau dapat mengatakan hal-hal ini?'' 
\end{quote}
Vase\d{t}\d{t}hī Therī:
\begin{quote}
`Brahmana, Yang Tercerahkan\\
 Di kota Mithilā,\\
 Mengajarkan Dhamma sehingga makhluk-makhluk hidup\\
 Dapat meninggalkan segala penderitaan.

Setelah mendengarkan ajaran dari Yang Sempurna,\\
 Brahmana, yang bebas dari segala kemelekatan,\\
 Setelah memahami ajaran sejati di sana,\\
 Aku menyingkirkan kesedihan atas anak-anakku.'' 
\end{quote}
Brāhma\d{n}a Sujāta (ayahnya Sundarī Therī) bertanya kepada Vāse\d{t}\d{t}hi
Therī:
\begin{quote}
``Aku juga akan pergi\\
 Ke kota Mithilā.\\
 Semoga Sang Buddha dapat membebaskan aku\\
 Dari segala penderitaan.'' 
\end{quote}
Penyusun:
\begin{quote}
Sang brahmana melihat Sang Buddha,\\
 Yang terbebaskan, tanpa kemelekatan.\\
 Beliau mengajarkan Dhamma kepadanya,\\
 Sang bijaksana yang melampaui penderitaan:

Penderitaan, asal-mula penderitaan,\\
 Melampaui penderitaan,\\
 Dan Jalan Mulia Berunsur Delapan\\
 Yang mengarah menuju diamnya penderitaan.

Setelah memahami ajaran sejati di sana,\\
 Ia setuju untuk meninggalkan keduniawian.\\
 Tiga hari kemudian\\
 Sujāta merealisasikan tiga pengetahuan. 
\end{quote}
Brāhma\d{n}a Sujāta (ayahnya Sundarī Therī) kepada kusirnya:
\begin{quote}
``Mohon, kusir, pergilah;\\
 Bawa kereta ini kembali.\\
 Dengan mengharapkan kesehatan nyonya brahmana, katakan:\\
 `Sang brahmana sekarang telah meninggalkan keduniawian.\\
 Setelah tiga hari,\\
 Sujāta merealisasikan tiga pengetahuan.'\,'' 
\end{quote}
Penyusun:
\begin{quote}
Kemudian dengan membawa kereta,\\
 beserta seribu keping uang, sang kusir\\
 Mengharapkan kesehatan nyonya brahmana, dan berkata:\\
 ``Sang brahmana sekarang telah meninggalkan keduniawian.\\
 Setelah tiga hari,\\
 Sujātā merealisasikan tiga pengetahuan.'' 
\end{quote}
Ibunya Sundarī Therī kepada sang kusir:
\begin{quote}
Mendengar bahwa sang brahmana telah memiliki tiga pengetahuan, sang
nyonya berkata:\\
 ``Aku memberikan kepadamu kuda dan kereta ini,\\
 O Kusir, beserta 1000 keping uang,\\
 Dan sebuah mangkuk yang penuh ini sebagai hadiah.'' 
\end{quote}
Kusir kepada ibunya Sundarī Therī:
\begin{quote}
``Simpanlah kuda dan kereta ini, nyonya,\\
 Beserta seribu keping uang ini.\\
 Aku juga akan meninggalkan keduniawian di hadapan Beliau,\\
 Orang ini memiliki kebijaksanaan yang mengagumkan.'' 
\end{quote}
Ibunya Sundarī Therī kepada Sundarī Therī:
\begin{quote}
``Gajah-gajah, sapi, perhiasan and anting-anting,\\
 Kekayaan rumah-tangga yang mewah demikian:\\
 Setelah melepaskannya, ayahmu meninggalkan keduniawian,\\
 Nikmatilah kekayaan ini Sundarī,\\
 Engkau adalah pewaris keluarga.'' 
\end{quote}
Sundarī Therī kepada ibunya:
\begin{quote}
``Gajah-gajah, sapi, perhiasan and anting-anting,\\
 Kekayaan rumah-tangga yang mewah demikian:\\
 Setelah melepaskannya, ayahku meninggalkan keduniawian,\\
 Karena tersiksa oleh kesedihan atas putranya.\\
 Aku juga akan meninggalkan keduniawian,\\
 Karena tersiksa oleh kesedihan atas saudara laki-lakiku.'' 
\end{quote}
Penahbis Sundarī Therī kepada Sundarī Therī:
\begin{quote}
``Sundarī, semoga harapan yang engkau inginkan\\
 Menjadi kenyataan.\\
 Dengan mengumpulkan sisa makanan sedikit demi sedikit,\\
 Dan kain-kain buangan sebagai jubah---\\
 Dengan memanfaatkan benda-benda ini,\\
 Bebas dari kekotoran sehubungan dengan kehidupan mendatang.'' 
\end{quote}
Sundarī Therī kepada penahbisnya:
\begin{quote}
``Nyonya, sewaktu aku masih menjadi seorang calon bhikkhunī,\\
 Mata-dewaku menjadi murni;\\
 Aku mengetahui kehidupan-kehidupan lampauku,\\
 Tempat-tempat di mana dulu aku hidup.

Dengan mengandalkan seorang nyonya baik seperti engkau,\\
 Seorang bhikkhunī senior yang memperindah Sangha,\\
 Aku telah mencapai tiga pengetahuan,\\
 Dan memenuhi ajaran Sang Buddha.

Berilah izin kepadaku, nyonya,\\
 Aku ingin pergi ke Sāvatthī,\\
 Di mana aku akan mengaumkan auman singaku\\
 Di hadapan yang terbaik di antara para Buddha.'' 
\end{quote}
Penahbis Sundarī Therī kepada Sundarī Therī:
\begin{quote}
``Sundarī, temuilah Sang Guru!\\
 Yang berwarna keemasan, berkulit keemasan,\\
 Penjinak mereka yang belum jinak,\\
 Yang Tercerahkan yang tidak takut pada apapun dari segala penjuru.'' 
\end{quote}
Penyusun:
\begin{quote}
``Lihatlah Sundarī datang,\\
 Terbebaskan, tanpa kemelekatan;\\
 Tanpa keinginan, terlepas,\\
 Tugasnya selesai, tanpa kekotoran.''

``Setelah berangkat dari Bārā\d{n}asī\\
 Dan datang ke hadapanMu, Pahlawan Besar,\\
 siswiMu Sundarī\\
 bersujud di kakiMu.'' 
\end{quote}
Sundarī Therī:
\begin{quote}
``Engkau adalah Sang Buddha, Engkau adalah Sang Guru,\\
 Aku adalah putri sejatimu, Brahmana,\\
 Yang terlahir dari mulutMu.\\
 Aku telah menyelesaikan tugas dan bebas dari kekotoran.'' 

\newpage
\end{quote}
Sang Buddha:
\begin{quote}
``Kalau begitu selamat datang, nyonya yang baik,\\
 Engkau tidak mungkin tidak disambut.\\
 Karena ini adalah bagaimana mereka yang jinak datang\\
 Bersujud di kaki Sang Guru;\\
 Tanpa keinginan, terlepas,\\
 Tugas selesai, tanpa kekotoran.'' 
\end{quote}
\hypertarget{subha-putri-pandai-besi}{

\section{13.5 Subha, Putri Pandai Besi}

\label{subha-putri-pandai-besi}}

Subhā Therī, putri pandai besi:
\begin{quote}
``Aku masih begitu muda, pakaianku begitu segar,\\
 Pada waktu aku mendengar ajaran.\\
 Karena tekun,\\
 Aku memahami kebenaran;

Dan kemudian aku menjadi sangat bosan\\
 Pada segala kenikmatan indria\\
 Melihat dengan takut pada identitas\\
 Aku mendambakan pelepasan keduniawian.

Dengan meninggalkan lingkaran keluarga,\\
 Para pelayan dan pekerja,\\
 Dan desa serta lahan yang subur,\\
 Yang begitu indah dan menyenangkan,

Aku meninggalkan keduniawian;\\
 Semua itu bukan kekayaan yang kecil.\\
 Sekarang aku telah meninggalkan keduniawian dalam keyakinan seperti
ini,\\
 Dalam ajaran sejati yang dibabarkan dengan baik,

Karena aku berkeinginan untuk tidak memiliki apa-apa,\\
 Tidaklah selayaknya\\
 Untuk mengambil kembali emas dan uang,\\
 Setelah meninggalkannya.

Uang atau emas\\
 Tidak mengarah menuju kedamaian dan pencerahan.\\
 Itu tidak selayaknya bagi seorang petapa,\\
 Itu bukanlah kekayaan para mulia;

Itu hanyalah keserakahan dan kemabukan,\\
 Kebingungan dan kemerosotan yang meningkat,\\
 Meragukan, menyulitkan---\\
 Tidak ada yang bertahan lama di sana.

Bejat dan lengah,\\
 Orang-orang yang tak tercerahkan, batinnya rusak\\
 Saling melawan satu sama lain,\\
 Menciptakan perselisihan.

Pembunuhan, pengurungan, kesengsaraan,\\
 Kehilangan, kesedihan, dan ratapan;\\
 Mereka yang tenggelam dalam kenikmatan indria\\
 Melihat banyak bencana.

Keluargaku, mengapa kalian mendorongku\\
 Pada kenikmatan, seolah-olah kalian adalah musuhku?\\
 Kalian tahu aku telah meninggalkan keduniawian,\\
 Melihat dengan takut pada kenikmatan indria.

Bukan karena emas, dalam bentuk kepingan uang atau bukan kepingan
uang,\\
 Kekotoran-kekotoran itu berakhir.\\
 Kenikmatan indria adalah musuh dan pembunuh,\\
 Kekuatan jahat yang mengikat engkau pada duri.

Keluargaku, mengapa kalian mendorongku\\
 Pada kenikmatan, seolah-olah kalian adalah musuhku?\\
 Kalian tahu aku telah meninggalkan keduniawian,\\
 Dengan kepala tercukur, terbalut dalam jubah luarku.

Mengumpulkan sisa makanan sedikit demi sedikit,\\
 Dan kain buangan sebagai jubah---\\
 Itu adalah apa yang layak bagiku,\\
 Inti dari kehidupan tanpa rumah.

Para petapa besar menghalau kenikmatan indria,\\
 Baik manusiawi maupun surgawi.\\
 Aman dalam suaka mereka, mereka terbebaskan,\\
 Setelah menemukan kebahagiaan yang tak tergoyahkan.

Semoga aku tidak menemukan kenikmatan indria,\\
 Karena tidak ada naungan di dalamnya.\\
 Kenikmatan indria adalah musuh dan pembunuh,\\
 Yang sama menyakitkannya seperti api unggun.

Keserakahan adalah sebuah rintangan, sebuah ancaman,\\
 Penuh kesedihan dan duri;\\
 Yang tidak seimbang,\\
 Jalan besar menuju kebingungan.

Berbahaya dan menakutkan,\\
 Kenikmatan indria adalah bagaikan kepala ular,\\
 Di mana orang-orang dungu bersenang,\\
 Orang-orang biasa yang terjebak dalam kegelapan.

Terjebak dalam lumpur kenikmatan indria,\\
 Ada begitu banyak orang dungu di dunia.\\
 Mereka tidak mengetahui akhir\\
 Dari kelahiran kembali dan kematian.

Karena kenikmatan indria,\\
 Orang-orang melompat ke jalan yang membawa mereka ke tempat yang
buruk.\\
 Begitu banyak yang berjalan di jalan itu\\
 Yang membawa penyakit pada diri mereka.

Demikianlah bagaimana kenikmatan menciptakan musuh-musuh;\\
 Mereka begitu menyiksa, begitu merusak,\\
 Yang menjebak makhluk-makhluk dengan kenikmatan materi duniawi,\\
 Yang bukan lain adalah belenggu kematian.

Menggilakan, menggoda,\\
 Kenikmatan indria merusak pikiran,\\
 Itu adalah jerat yang dipasang oleh Māra\\
 Untuk merusak makhluk-makhluk.

Kenikmatan indria adalah berbahaya tak terhingga,\\
 Penuh penderitaan, racun yang mengerikan;\\
 Menawarkan sedikit kepuasan, pembuat perselisihan,\\
 Melenyapkan kualitas-kualitas cerah.

Karena aku telah menciptakan begitu banyak kerusakan\\
 Karena kenikmatan indria,\\
 Aku tidak akan berpaling kembali kepadanya lagi,\\
 Melainkan akan selalu bersenang dalam pemadaman.

Bertarung melawan kenikmatan indria,\\
 Karena mendambakan keadaan sejuk itu,\\
 Aku akan bermeditasi dengan tekun\\
 Untuk mengakhiri segala belenggu.

Tanpa dukacita, tanpa noda, aman:\\
 Aku akan mengikuti jalan itu,\\
 Jalan Mulia Berunsur Delapan yang lurus\\
 Yang dengannya para petapa telah menyeberang.'' 
\end{quote}
Sang Buddha:
\begin{quote}
``Lihatlah ini: Subhā putri pandai besi,\\
 Berdiri tegak dalam ajaran.\\
 Ia telah memasuki keadaan tanpa gangguan,\\
 Dengan bermeditasi di bawah pohon.

Hanya delapan hari sejak ia meninggalkan keduniawian,\\
 Penuh keyakinan dalam ajaran yang indah.\\
 Dibimbing oleh Uppalava\d{n}\d{n}ā,\\
 Ia adalah penguasa tiga pengetahuan, penghancur kematian.

Orang ini terbebaskan dari pembudakan dan utang,\\
 Seorang bhikkhunī dengan indria-indria terkembang.\\
 Terlepas dari segala kemelekatan,\\
 Ia telah menyelesaikan tugas dan bebas dari kekotoran.'' 
\end{quote}
Penyusun:
\begin{quote}
Demikianlah Sakka, raja semua makhluk,\\
 Bersama dengan kumpulan para dewa,\\
 Setelah datang dengan kekuatan batin mereka,\\
 Memberi hormat kepada Subhā, sang putri pandai besi. 
\end{quote}
Kelompok Dua Puluh selesai.

\hypertarget{subhux101-dari-hutan-mangga-jux12bvaka}{

\newpage
\chapter{Kelompok Tiga Puluh}

\section{14.1 Subhā dari Hutan Mangga Jīvaka}

\label{subhux101-dari-hutan-mangga-jux12bvaka}}
\begin{quote}
Pergi ke hutan mangga yang indah\\
 Milik Jīvaka, bhikkhunī Subhā\\
 Dihalangi oleh seorang penjahat.\\
 Subhā mengatakan ini kepadanya: 
\end{quote}
Subhā Jīvakambavanikā Therī:
\begin{quote}
``Kesalahan apa yang telah kuperbuat kepadamu,\\
 Sehingga engkau menghalangi jalanku?\\
 Tuan, tidaklah selayaknya seorang laki-laki\\
 Menyentuh seorang perempuan yang meninggalkan keduniawian.

Latihan ini diajarkan oleh Yang Suci,\\
 Ini adalah persoalan serius dalam ajaran Guruku.\\
 Aku murni dan bebas dari noda,\\
 Jadi mengapa engkau menghalangi jalanku?

Seorang yang pikirannya ternoda kepada seorang yang tanpa noda;\\
 Seorang yang bernafsu kepada seorang yang bebas dari nafsu;\\
 Tanpa noda, batinku terbebaskan dalam segala aspek,\\
 Jadi mengapa engkau menghalangi jalanku?'' 

\newpage
\end{quote}
Penjahat:
\begin{quote}
``Engkau muda dan tanpa cacat---\\
 Apa gunanya pelepasan keduniawian bagimu?\\
 Buanglah jubah kuning itu,\\
 Mari bermain di taman bunga ini.

Di segala arah, aroma serbuk sari menguar manis,\\
 Yang berasal dari hutan berbunga.\\
 Awal musim semi adalah hari yang membahagiakan---\\
 Mari bermain di taman bunga ini.

Dan pepohonan bermahkotakan bunga-bunga\\
 Seolah-olah berbisik dalam tiupan angin.\\
 Tetapi kesenangan apakah yang akan engkau peroleh\\
 Dengan memasuki hutan ini sendirian?

Yang sering dikunjungi oleh gerombolan pemangsa,\\
 Dan gajah-gajah betina yang terangsang para pejantan pada musim kawin;\\
 Engkau ingin pergi tanpa teman\\
 Menuju hutan yang sepi dan menggetarkan.

Bagaikan boneka emas yang berkilau,\\
 Bagaikan bidadari yang mengembara di taman dengan tanaman berwarna-warni,\\
 Kecantikanmu yang tanpa tandingan akan bersinar.\\
 Dalam balutan pakaian dari kain kasa yang indah.

Aku akan menuruti perintahmu,\\
 Jika kita menetap di hutan ini.\\
 Aku tidak mencintai makhluk lain selain engkau,\\
 Oh peri dengan mata yang begitu indah.

Jika engkau menerima undanganku---\\
 `Mari, berbahagialah, dan menetap di sebuah rumah'---\\
 Engkau akan berdiam dalam sebuah rumah panjang yang terlindung dari
angin;\\
 Biarkan para gadis mengurus kebutuhanmu.

Berpakaian dari kain kasa indah,\\
 Kenakanlah kalung bunga dan kosmetik.\\
 Aku akan menyediakan segala jenis perhiasan untukmu,\\
 Dan emas dan permata dan mutiara.

Naiklah ke atas tempat tidur mahal,\\
 Penutupnya begitu bersih dan indah,\\
 Dengan matras wol yang baru,\\
 Begitu harum, dengan percikan cendana.

Bagaikan bunga bakung biru yang keluar dari air\\
 Tetap tidak tersentuh oleh manusia,\\
 Demikian pula, O nyonya yang murni dan suci,\\
 Tangan dan kakimu menua tanpa pemilik.'' 
\end{quote}
Subhā Jīvakambavanikā Therī:
\begin{quote}
``Tubuh ini adalah bangkai, memenuhi\\
 Tanah pemakaman, karena sifatnya yang hancur berserakan.\\
 Apa menurutmu yang menjadi intinya\\
 Sehingga engkau menatapku dengan tergila-gila?'' 
\end{quote}
Penjahat:
\begin{quote}
``Matamu seperti mata rusa betina,\\
 Atau peri di pegunungan;\\
 Melihatnya\\
 Keinginan indriaku semakin bertambah.

Tegakkan wajahmu yang tanpa cacat dan berkilau keemasan\\
 Matamu bagaikan kuntum bunga bakung biru;\\
 Melihatnya\\
 Keinginan indriaku semakin bertambah.

Walaupun engkau mungkin mengembara jauh, aku tetap memikirkan engkau,\\
 Dengan bulu matamu yang panjang, dan penglihatanmu begitu jernih.\\
 Aku tidak menyukai mata lain selain matamu,\\
 O peri dengan mata yang begitu indah.'' 

\newpage
\end{quote}
Subhā Jīvakambavanikā Therī:
\begin{quote}
``Engkau berjalan di jalan yang salah!\\
 Engkau ingin mengambil bulan sebagai mainanmu!\\
 Engkau mencoba untuk melompati Gunung Meru!\\
 Engkau, yang sedang berburu anak Sang Buddha!

Karena di dunia ini bersama dengan seluruh para dewanya,\\
 Tidak akan ada lagi nafsu di manapun padaku.\\
 Aku bahkan tidak tahu apa itu,\\
 Itu telah menjadi akar yang hancur dan semuanya berkat sang jalan.

Terlempar bagaikan percikan dari bara api,\\
 Bernilai tidak lebih dari semangkuk racun.\\
 Aku bahkan tidak tahu apa itu,\\
 Itu telah menjadi akar yang hancur dan semuanya berkat sang jalan.

Baiklah engkau boleh mencoba merayu jenis perempuan\\
 Yang belum merefleksikan hal-hal ini,\\
 Atau yang belum pernah melayani Sang Guru:\\
 Tetapi {*}ini{*} adalah perempuan yang mengetahui---sekarang engkau
mendapat masalah!

Tidak peduli apakah aku dihina atau dipuji,\\
 Atau merasa senang atau sakit: aku tetap penuh perhatian.\\
 Karena mengetahui bahwa kondisi-kondisi adalah buruk,\\
 Batinku tidak melekat pada apapun.

Aku adalah siswi dari Yang Suci,\\
 Berkendara dalam kereta jalan berunsur delapan.\\
 Anak panah tercabut, bebas dari kekotoran,\\
 Aku bahagia telah sampai di tempat kosong.

Aku telah melihat boneka dan wayang\\
 Yang diwarnai cerah\\
 Diikat pada tongkat dan tali,\\
 Dan dibuat menari dalam berbagai cara.

Tetapi ketika tongkat dan tali itu dilepaskan---\\
 Kendur, terlepas, terbongkar,\\
 Tidak dapat dipasang lagi, bagian-bagiannya terlucuti---\\
 Pada apakah pikiran ini tertambat?

itu adalah seperti apa adanya tubuhku,\\
 tanpa hal-hal itu tubuh ini tidak dapat berlanjut.\\
 Oleh karena itu,\\
 Pada apakah pikiran ini tertambat?

Ini seperti ketika engkau melihat lukisan dinding,\\
 Dilukis dengan pewarna kuning,\\
 Dan penglihatanmu menjadi buram,\\
 Secara keliru melihat bahwa itu adalah orang.

Walaupun ini tidak bernilai bagaikan tipuan sulap,\\
 Atau sebatang pohon emas yang terlihat dalam mimpi,\\
 Engkau secara membuta mengejar apa yang kosong,\\
 Bagaikan pertunjukan wayang di antara orang-orang.

Sebuah biji mata hanyalah sebuah bola dalam rongganya\\
 Dengan pupil mata di tengah, dan air mata,\\
 Dan lendir keluar dari sana juga,\\
 Demikianlah bagian-bagian mata berbeda berkumpul menjadi satu.'' 
\end{quote}
Penyusun:
\begin{quote}
Nyonya cantik itu mencabut keluar bola matanya.\\
 Dengan sama sekali tanpa kemelekatan dalam batinnya, ia berkata:\\
 ``Kemarilah, ambil mata ini,''\\
 Dan memberikannya kepada laki-laki itu di sana pada saat itu juga. 
\end{quote}
Penjahat:
\begin{quote}
Pada pada momen itu ia kehilangan nafsunya,\\
 Dan meminta maaf:\\
 ``Semoga engkau baik, O Nyonya yang murni dan suci;\\
 Ini tidak akan terjadi lagi.

Menyerang seorang seperti ini\\
 Adalah bagaikan menggenggam kobaran api,\\
 Atau menangkap ular berbisa yang mematikan!\\
 Semoga engkau baik, mohon maafkan aku.'' 

\newpage
\end{quote}
Penyusun:
\begin{quote}
Ketika bhikkhunī itu dilepaskan\\
 Ia pergi menghadap Sang Buddha yang agung.\\
 Melihat seorang yang memiliki tanda jasa yang unggul,\\
 Matanya pulih kembali seperti semula. 
\end{quote}
Kelompok Tiga Puluh selesai.

\hypertarget{isidux101sux12b}{

\newpage
\chapter{Kelompok Umpat Puluh}

\section{15.1 Isidāsī}

\label{isidux101sux12b}}

Penyusun:
\begin{quote}
Di Pā\d{t}aliputta, yang terbaik di dunia,\\
 Kota yang dinamai dari nama bunga,\\
 Ada dua bhikkhunī dari suku Sakya,\\
 Keduanya adalah perempuan berkualitas.

Satu bernama Isidāsī, yang ke dua bernama Bodhī.\\
 Keduanya sempurna dalam etika,\\
 Menyukai meditasi dan melafal,\\
 Terpelajar, menggilas kerusakan.

Mereka berkeliling menerima dana makan dan memakan makanan mereka.\\
 Setelah mereka mencuci mangkuk mereka,\\
 Mereka duduk dengan bahagia di tempat sunyi\\
 Dan memulai percakapan. 
\end{quote}
Bodhī Therī:
\begin{quote}
``Engkau sangat cantik, Yang Mulia Isidāsī\\
 Kemudaanmu belum memudar.\\
 Masalah apakah yang engkau lihat sehingga engkau\\
 Mengabdikan diri pada pelepasan keduniawian?'' 

\newpage
\end{quote}
Isidāsī Therī:
\begin{quote}
Ditanya seperti ini secara pribadi,\\
 Isidāsī, yang terampil dalam membabarkan Dhamma,\\
 Menjawab dengan kata-kata sebagai berikut,\\
 ``Bodhī, dengarkanlah bagaimana aku meninggalkan keduniawian.

Di kota Ujjenī yang indah,\\
 Ayahku adalah seorang hartawan, seorang yang baik dan bermoral.\\
 Aku adalah putri tunggalnya,\\
 Yang disayangi, dicintai, dan dikasihi.

Kemudian beberapa peminang datang untuk meminangku\\
 Dari keluarga terpandang di Sāketa.\\
 Mereka diutus oleh seorang hartawan dengan kekayaan berlimpah,\\
 Yang padanya ayahku kemudian menyerahkan aku sebagai menantu.

Setiap pagi dan malam hari,\\
 Aku bersujud dengan kepala di kaki\\
 Ayah dan ibu mertuaku,\\
 Seperti yang dikatakan kepadaku.

Kapanpun aku melihat saudari-saudari suamiku,\\
 Saudara-saudaranya, para pelayannya,\\
 Atau bahkan ia, milikku satu-satunya,\\
 Aku dengan gugup menyiapkan tempat duduk untuk mereka.

Apapun yang mereka inginkan---makanan dan minuman,\\
 Kudapan, atau apapun yang ada di lemari---\\
 Aku membawanya keluar dan menawarkan kepada mereka,\\
 Memastikan masing-masing dari mereka mendapatkan apa yang selayaknya.

Setelah bangun pagi,\\
 Aku mendatangi rumah utama,\\
 Mencuci tangan dan kakiku,\\
 Dan menghadap suamiku dengan merangkapkan tangan.

Dengan mengambil sisir, perhiasan,\\
 Penghitam alis, dan cermin,\\
 Aku merias sendiri suamiku,\\
 Seolah-olah aku adalah juru riasnya.

Aku sendiri yang memasak nasi;\\
 Aku sendiri yang mencuci panci.\\
 Aku merawat suamiku\\
 Bagaikan seorang ibu merawat anak tunggalnya.

Demikianlah aku memperlihatkan pengabdian kepadanya,\\
 Seorang pelayan yang mencintai, bermoral, dan sederhana,\\
 Bangun pagi, dan bekerja tanpa kenal lelah:\\
 Namun suamiku masih menyalahkan aku.

Ia berkata kepada ibu dan ayahnya:\\
 ``Aku akan pergi,\\
 Aku tidak tahun hidup bersama Isidāsī\\
 Menetap di rumah yang sama.''

``Anakku, jangan berkata seperti itu!\\
 Isidāsi cerdik dan kompeten,\\
 Ia bangun pagi dan bekerja tanpa kenal lelah,\\
 Anakku, mengapa ia tidak menyenangkan engkau?''

``Ia tidak melakukan apapun yang menyakiti aku,\\
 Tetapi aku hanya tidak tahan hidup bersamanya.\\
 Sejauh yang aku tahu, ia hanya mengerikan.\\
 Cukup sudah, aku akan pergi.''

Ketika mereka mendengar kata-katanya,\\
 Ayah dan ibu mertuaku bertanya kepadaku:\\
 ``Kesalahan apakah yang engkau perbuat?\\
 Katakanlah kepada kami dengan jujur, jangan takut.''

``Aku tidak melakukan kesalahan apapun,\\
 Aku tidak pernah menyakitinya, atau mengatakan apapun yang buruk.\\
 Apa yang mungkin kulakukan,\\
 Ketika suamiku begitu membenciku?''

Mereka membawaku pulang ke rumah ayahku,\\
 Dengan putus asa, dikuasai oleh penderitaan, dan berkata:\\
 ``Karena menyayangi putra kami,\\
 Kami kehilangannya, yang begitu cantik dan beruntung!''

\newpage

Selanjutnya ayahku menyerahkan aku ke rumah tangga\\
 Laki-laki dari keluarga kaya ke dua.\\
 Untuk ini ia memperoleh setengah harga-pengantin\\
 Dari apa yang dibayarkan oleh si hartawan.

Di rumahnya aku juga menetap selama satu bulan,\\
 Sebelum ia juga menginginkan aku pergi;\\
 Walaupun aku melayaninya bagaikan budak,\\
 Baik dan tidak melakukan kesalahan.

Ayahku kemudian berkata kepada seorang pengemis yang meminta makanan,\\
 Seorang penjinak orang lain dan dirinya sendiri:\\
 ``Jadilah menantuku;\\
 Singkirkan jubah dan mangkukmu.''

Ia menetap selama dua minggu sebelum ia berkata kepada ayahku:\\
 ``Kembalikan jubah usangku,\\
 Mangkukku, dan cangkirku---\\
 Aku akan berkeliling mengemis makanan lagi.''

Maka kemudian ibu dan ayahku\\
 Dan seluruh sanak saudaraku berkata:\\
 ``Apa yang belum dilakukan untukmu di sini?\\
 Cepatlah, katakan apa yang dapat kami lakukan untukmu!''

Ketika mereka berkata kepadanya seperti ini ia berkata,\\
 ``Bahkan jika engkau menyembahku, cukup sudah.\\
 Aku tidak tahan hidup bersama Isidāsī\\
 Menetap di rumah yang sama.''

Setelah dibebaskan, ia pergi.\\
 Tetapi aku duduk sendirian merenungkan:\\
 ``Setelah meminta izin, aku akan pergi,\\
 Apakah mati atau meninggalkan keduniawian.''

Tetapi kemudian Yang Mulia Nyonya Jinadattā,\\
 Yang terpelajar dan bermoral,\\
 Yang telah menghafalkan teks latihan monastik,\\
 Mendatangi rumah ayahku untuk menerima dana makanan.

\newpage

Ketika aku melihatnya,\\
 Aku bangkit dari dudukku dan mempersiapkan tempat duduk itu untuknya.\\
 Ketika ia telah duduk,\\
 Aku bersujud di kakinya dan mempersembahkan makanan kepadanya,

Memuaskannya dengan makanan dan minuman,\\
 Kudapan, dan apapun yang ada dalam lemari.\\
 Kemudian aku berkata:\\
 ``Nyonya, aku ingin meninggalkan keduniawian!''

Tetapi ayahku berkata kepadaku:\\
 ``Anakku, berlatihlah Dhamma di sini!\\
 Persembahkan para petapa dan para brahmana yang terlahir dua kali\\
 Dengan makanan dan minuman.''

Kemudian aku berkata kepada ayahku,\\
 Sambil menangis, merangkapkan tanganku ke arahnya:\\
 ``Aku telah melakukan hal buruk di masa lampau;\\
 Aku harus menebus perbuatan buruk itu.''

Dan ayahku berkata kepadaku:\\
 ``Semoga engkau mencapai pencerahan, keadaan tertinggi,\\
 Dan semoga engkau menemukan pemadaman\\
 Yang direalisasikan oleh orang-orang terbaik!''

Aku bersujud kepada ibu dan ayahku,\\
 Dan seluruh sanak-saudaraku;\\
 Dan kemudian, tujuh hari setelah pelepasan keduniawian,\\
 Aku merealisasikan tiga pengetahuan.

Aku mengetahui tujuh kehidupan terakhirku;\\
 Aku akan menceritakan perbuatan-perbuatannya kepadamu\\
 Yang mana buah dan akibatnya terjadi dalam kehidupan ini:\\
 Arahkan seluruh perhatianmu pada kisah ini.

Di kota Erakacca\\
 Aku adalah seorang pandai emas yang memiliki banyak uang.\\
 Mabuk karena bangga atas kemudaanku,\\
 Aku berhubungan intim dengan istri orang lain.

\newpage

Setelah meninggal dunia dari sana,\\
 Aku dibakar dalam neraka untuk waktu yang lama.\\
 Keluar dari sana\\
 Aku dikandung dalam rahim seekor monyet.

Ketika aku baru berusia tujuh hari,\\
 Aku dikebiri oleh pemimpin monyet.\\
 Ini adalah buah dari perbuatan itu,\\
 Karena perselingkuhan dengan istri orang lain.

Setelah meninggal dunia dari sana,\\
 Meninggal dunia di hutan Sindhava,\\
 Aku dikandung dalam rahim\\
 Seekor kambing betina yang cacat, bermata-satu.

Aku membawa anak-anak di punggungku selama dua belas tahun,\\
 Dan selama itu aku dikebiri,\\
 Dirubungi belatung, dan tanpa ekor,\\
 Karena perselingkuhan dengan istri orang lain.

Setelah meninggal dunia dari sana,\\
 Aku terlahir kembali sebagai seekor sapi\\
 Yang dimiliki oleh seorang pedagang sapi.\\
 Seekor anak sapi merah, dikebiri, selama dua belas bulan.

Aku menarik bajak besar.\\
 Aku memikul kereta,\\
 Buta, tanpa ekor, lemah,\\
 Karena perselingkuhan dengan istri orang lain.

Setelah meninggal dunia dari sana,\\
 Aku terlahir sebagai pelacur di jalan,\\
 Bukan perempuan juga bukan laki-laki,\\
 Karena perselingkuhan dengan istri orang lain.

Aku meninggal dunia pada usia tiga puluh,\\
 Dan terlahir kembali sebagai seorang gadis dalam keluarga pembuat
kereta.\\
 Kami miskin, dengan sedikit harta,\\
 Sangat tertindas oleh kreditor.

\newpage

Karena bunga pinjaman yang sangat besar,\\
 Aku diseret sambil menangis,\\
 Dibawa paksa dari rumah keluarga\\
 Oleh pemimpin karavan

Ketika aku berusia enam belas tahun,\\
 Putranya yang bernama Giridāsa,\\
 Melihat bahwa aku adalah seorang gadis yang telah cukup usia untuk
menikah,\\
 Mengambilku sebagai istrinya.

Ia juga memiliki seorang istri lainnya,\\
 Seorang nyonya berkualitas yang terkenal dan bermoral,\\
 Setia kepada suaminya;\\
 Tetapi aku memicu kebenciannya.

Sebagai buah dari perbuatan itu,\\
 Mereka meninggalkan aku dan pergi,\\
 Walaupun aku melayani mereka bagaikan budak.\\
 Sekarang aku telah mengakhiri ini juga.'' 
\end{quote}
Kelompok Empat Puluh selesai.

\hypertarget{sumedhux101}{

\newpage
\chapter{Kelompok Panjang}

\section{16.1 Sumedhā}

\label{sumedhux101}}
\begin{quote}
Di kota Mantāvatī, Sumedhā,\\
 Putri dari permaisuri Raja Koñca,\\
 Dialihkan keyakinannya oleh mereka\\
 Yang berlatih ajaran Sang Buddha.

Ia bermoral, seorang pembabar yang cemerlang,\\
 Terpelajar, dan terlatih dalam ajaran-ajaran Sang Buddha.\\
 Ia mendatangi ibu dan ayahnya dan berkata:\\
 ``Perhatikanlah, kalian berdua!

Aku bersenang dalam pemadaman!\\
 Tidak ada kehidupan yang abadi, bahkan tidak kehidupan para dewa;\\
 Apalagi kenikmatan indria, begitu kosong,\\
 Memberikan sedikit kepuasan dan banyak kesusahan.

Kenikmatan indria adalah pahit bagaikan bisa ular,\\
 Namun orang-orang dungu tergila-gila padanya.\\
 Dikirim ke neraka untuk waktu yang sangat lama,\\
 Mereka dipukul dan disiksa.

Mereka yang tumbuh dalam kejahatan\\
 Selalu berduka di alam bawah karena perbuatan jahat mereka sendiri.\\
 Mereka adalah orang-orang dungu, tidak terkekang dalam jasmani,\\
 Pikiran, dan ucapan.

\newpage

Mereka orang-orang dungu yang bodoh dan tidak bijaksana,\\
 Dihalangi oleh asal-mula penderitaan,\\
 Bodoh, tidak memahami kebenaran-kebenaran mulia\\
 Ketika mereka sedang diajarkan.

Kebanyakan orang, ibu, tidak mengetahui kebenaran-kebenaran\\
 Yang diajarkan oleh Sang Buddha yang agung,\\
 Mengharapkan kehidupan berikut,\\
 Mendambakan kelahiran kembali di antara para dewa.

Akan tetapi bahkan kelahiran kembali di antara para dewa\\
 Dalam keadaan tidak kekal adalah tidak abadi.\\
 Tetapi orang-orang dungu tidak takut\\
 Pada kelahiran kembali yang berulang-ulang.

Empat alam rendah dan dua alam lainnya\\
 Dapat dicapai dalam suatu cara atau lainnya.\\
 Tetapi bagi mereka yang berakhir di alam rendah,\\
 Tidak ada cara untuk meninggalkan keduniawian di neraka.

Sudilah kalian berdua memberiku izin untuk meninggalkan keduniawian\\
 Dalam pengajaran dari Beliau yang memiliki sepuluh kekuatan.\\
 Dengan hidup nyaman, aku akan mengerahkan diriku\\
 Untuk meninggalkan kelahiran kembali dan kematian.

Apa gunanya harapan, dalam kehidupan baru,\\
 Dalam tubuh yang kosong dan tidak berguna ini?\\
 Izinkanlah aku untuk meninggalkan keduniawian\\
 Untuk mengakhiri ketagihan pada kehidupan baru.

Seorang Buddha telah muncul, waktunya telah tiba,\\
 Saat tidak beruntung telah berlalu.\\
 Seumur hidup aku tidak akan pernah mengkhianati\\
 Aturan etika atau kehidupan selibatku.''

Kemudian Sumedhā berkata kepada orangtuanya:\\
 ``Selama aku masih menjadi orang awam,\\
 Aku akan menolak untuk memakan makanan apapun,\\
 Hingga aku jatuh di bawah kekuasaan kematian.''

Karena sedih, ibunya menangis,\\
 Sedangkan ayahnya, walaupun sedih,\\
 Berusaha keras untuk membujuknya\\
 Selagi ia roboh di lantai rumah panjang itu.

``Bangunlah anakku, mengapa engkau begitu bersedih?\\
 Engkau telah ditunangkan dan segera menikah!\\
 Raja Anīkaratta yang tampan\\
 Sedang berada di Vāra\d{n}avatī: ia adalah tunanganmu.

Engkau akan menjadi permaisuri,\\
 Istri dari Raja Anīkaratta.\\
 Aturan-aturan etika, kehidupan selibat---\\
 Pelepasan keduniawian adalah sulit dilakukan, anakku.

Sebagai anggota kerajaan ada perintah, kekayaan, kekuasaan,\\
 Dan kebahagiaan memiliki.\\
 Nikmatilah kenikmatan indria selagi engkau masih muda!\\
 Wujudkanlah pernikahanmu, anakku!''

Kemudian Sumedhā berkata kepadanya:\\
 ``Jangan sampai ini terjadi! Kehidupan adalah kosong!\\
 Aku akan meninggalkan keduniawian atau mati,\\
 Tetapi aku tidak akan pernah menikah.

Mengapa melekati tubuh busuk ini yang sangat kotor,\\
 Cairan busuk,\\
 Kantong-air mayat yang mengerikan,\\
 Selalu menetes, penuh kotoran?

Mengetahuinya seperti yang kuketahui, apa gunanya?\\
 Bangkai keji, yang dilumuri dengan daging dan darah,\\
 Makanan bagi burung-burung dan kawanan belatung---\\
 Mengapa kita diberikan ini?

Tidak lama kemudian tubuh ini, tanpa kesadaran,\\
 Dibawa ke tanah pemakaman,\\
 Untuk dibuang bagaikan anjing tua\\
 Oleh sanak-saudara dalam kejijikan.

Ketika mereka telah membuangnya di tanah pemakaman,\\
 Untuk dimakan oleh makhluk-makhluk lain, orangtuamu sendiri\\
 Mandi, dengan jijik:\\
 Apalagi orang-orang lainnya?

Mereka melekat pada bangkai kosong ini,\\
 Kumpulan urat dan tulang ini;\\
 Tubuh membusuk ini\\
 Penuh liur, air mata, tinja, dan nanah.

Jika seseorang membedahnya,\\
 Mengeluarkan isinya,\\
 Bau yang tak tertahankan\\
 Akan menjijikkan bahkan bagi ibunya sendiri.

Dengan seksama memeriksa\\
 Agregat-agregat, elemen-elemen, dan bidang-bidang indria\\
 Sebagai terkondisi, berakar dalam kelahiran, penderitaan---\\
 Mengapa aku menginginkan pernikahan?

Biarlah tiga ratus pedang tajam\\
 Menghujam tubuhku setiap hari!\\
 Bahkan jika pembantaian itu berlangsung 100 tahun\\
 Itu setimpal jika dapat mengarah menuju berakhirnya penderitaan

Seorang yang memahami kata-kata Sang Guru\\
 Akan menahankan pembantaian ini:\\
 `sungguh panjang bagimu transmigrasi ini\\
 Dengan terbunuh berulang-ulang.'

Di antara para dewa dan manusia,\\
 Di alam binatang dan alam siluman,\\
 Di antara para hantu dan mereka yang di neraka,\\
 Pembunuhan tanpa akhir terlihat.

Neraka dipenuhi pembunuhan,\\
 Bagi yang rusak yang telah jatuh ke alam bawah.\\
 Bahkan di antara para dewa tidak ada naungan,\\
 Karena tidak ada kebahagiaan yang melampaui pemadaman.

Mereka yang berkomitmen pada pengajaran\\
 Dari Beliau yang memiliki sepuluh kekuatan akan mencapai padamnya.\\
 Dengan hidup dalam kenyamanan, mereka mengerahkan diri\\
 Untuk meninggalkan kelahiran kembali dan kematian.

Pada hari ini juga, ayah, aku akan meninggalkan:\\
 Apakah yang dapat dinikmati dalam kekayaan hampa?\\
 Aku kecewa dengan kenikmatan indria,\\
 Yang seperti muntahan, dibuat menjadi seperti tunggul palem.''

Sewaktu ia berkata demikian kepada ayahnya,\\
 Anīkaratta, yang kepadanya ia ditunangkan,\\
 Datang dari Vāra\d{n}avatī\\
 Pada waktu yang telah ditentukan untuk menikah.

Kemudian Sumedhā mengambil pisau,\\
 Dan memotong rambutnya, yang begitu hitam, tebal, dan lembut.\\
 Dengan mengurung dirinya di dalam rumah panjang,\\
 Ia mencapai penyerapan pertama.

Dan ketika ia mencapainya di sana,\\
 Anīkaratta tiba di kota.\\
 Kemudian di dalam rumah panjang itu, Sumedhā\\
 Dengan baik mengembangkan persepsi ketidakkekalan.

Ketika ia menyelidiki dalam meditasi,\\
 Anīkaratta degan cepat menaiki tangga.\\
 Tangan dan kakinya berhiaskan permata dan emas,\\
 Ia memohon Sumedhā dengan merangkapkan tangan:

``Sebagai anggota kerajaan ada perintah, kekayaan, kekuasaan,\\
 Dan kebahagiaan memiliki.\\
 Nikmatilah kenikmatan indria selagi engkau masih muda!\\
 Kenikmatan indria adalah sulit diperoleh di dunia!

Aku menyerahkan kerajaan kepadamu---\\
 Nikmatilah kekayaan, berikan persembahan!\\
 Jangan bersedih;\\
 Orangtuamu sedih.''

Sumedhā, yang tidak menginginkan kenikmatan indria,\\
 Dan yang telah menyingkirkan delusi, menjawab:\\
 ``Jangan bersenang dalam indriawi!\\
 Lihatlah bahaya dalam kenikmatan indria!

Mandhātā, raja empat benua,\\
 Yang terkemuka dalam menikmati kenikmatan indria,\\
 Meninggal dunia tanpa terpuaskan,\\
 Keinginannya tidak terpenuhi.

Jika tujuh permata jatuh dari angkasa\\
 Di seluruh sepuluh penjuru,\\
 Tidak akan ada kepuasan dari kenikmatan indria:\\
 Orang-orang mati tanpa terpuaskan.

Bagaikan pisau penjagal dan alas pemotong,\\
 Kenikmatan indria adalah bagaikan kepala ular.\\
 Yang membakar bagaikan bara-api,\\
 Yang menyerupai kerangka.

Kenikmatan indria adalah tidak kekal dan tidak stabil,\\
 Penuh penderitaan, racun yang mengerikan;\\
 Bagaikan bola besi panas,\\
 Akar kesengsaraan, buahnya adalah kesakitan.

Kenikmatan indria adalah bagaikan buah dari pohon,\\
 Bagaikan bongkahan daging, menyakitkan,\\
 Yang memperdaya engkau bagaikan mimpi;\\
 Kenikmatan indria adalah bagaikan benda-benda pinjaman.

Kenikmatan indria adalah bagaikan pedang dan tombak;\\
 Penyakit, bisul, kesengsaraan dan kesulitan.\\
 Bagaikan lubang arang membara,\\
 Akar kesengsaraan, ketakutan dan pembantaian.

Demikianlah kenikmatan indria telah dijelaskan\\
 Sebagai halangan, begitu penuh penderitaan.\\
 Pergilah! Sehubungan denganku,\\
 Aku tidak mempercayai kehidupan baru.

Apa yang dapat dilakukan orang lain untukku\\
 Ketika kepala mereka sendiri sedang terbakar?\\
 Ketika diintai oleh penuaan dan kematian,\\
 Engkau harus berusaha untuk menghancurkannya.''

Ia membuka pintu\\
 Dan melihat orangtuanya bersama Anīkaratta,\\
 Duduk menangis di pintu.\\
 Maka ia mengatakan ini:

``Transmigrasi adalah lama bagi orang-orang dungu,\\
 Menangis berulang-ulang atas hal itu sejak awal yang tidak diketahui---\\
 Kematian ayah,\\
 Pembunuhan saudara atau diri mereka sendiri.

Ingatlah samudra air mata, susu, darah,\\
 Transmigrasi sejak awal yang tidak diketahui.\\
 Ingatlah tulang-belulang yang menggunung\\
 Oleh makhluk-makhluk yang bertransmigrasi.

Ingat empat samudra\\
 Yang dibandingkan dengan air mata, susu, dan darah;\\
 Ingat tulang-belulang yang menggunung setinggi Gunung Vipula\\
 Dalam perjalanan satu kappa.

Transmigrasi sejak awal yang tidak diketahui\\
 Dibandingkan dengan tanah India yang luas ini;\\
 Jika dibagi menjadi bongkahan berukuran biji buah jujube,\\
 Itu masih lebih sedikit daripada ibu dari ibunya.

Ingat rerumputan, tongkat kayu, dan dedaunan,\\
 Dibandingkan dengan awal yang tidak diketahui:\\
 Jika dipecahkan berkeping-keping berukuran empat inci,\\
 Itu masih lebih sedikit daripada ayahnya ayah.

Ingat kura-kura bermata satu dan kuk berlubang satu\\
 Yang hanyut di samudra dari timur ke barat---\\
 Memasukkan kepalanya ke dalam lubang\\
 Adalah perumpamaan untuk diperolehnya kelahiran sebagai manusia.

Ingat bentuk dari tubuh yang tidak beruntung ini,\\
 Tanpa inti bagaikan bongkahan buih.\\
 Lihatlah agregat-agregat sebagai tidak kekal,\\
 Ingatlah neraka yang penuh kesusahan.

Ingatlah tanah pemakaman yang menggunung itu\\
 Berulang-ulang dalam kehidupan demi kehidupan.\\
 Ingatlah ancaman buaya!\\
 Ingatlah empat kebenaran!

Ketika tanpa kematian ada untuk ditemukan,\\
 Mengapa engkau mau meminum lima racun pahit?\\
 Karena setiap kali menikmati kenikmatan indria\\
 Adalah jauh lebih pahit daripada itu.

Ketika tanpa kematian ada untuk ditemukan,\\
 Mengapa engkau mau terbakar demi kenikmatan indria?\\
 Karena setiap kali menikmati kenikmatan indria\\
 Adalah membakar, merebus, mendidih, menggelegak.

Ketika ada kebebasan dari permusuhan,\\
 Mengapa engkau menginginkan musuhmu, kenikmatan indria?\\
 Bagaikan raja-raja, api, para perampok, banjir, dan orang-orang yang
tidak engkau sukai,\\
 Kenikmatan indria adalah lebih dari musuhmu.

Ketika kebabasan ada untuk ditemukan,\\
 Apa gunanya kenikmatan indria yang membunuh dan mengikat?\\
 Karena walaupun tidak ingin, tetapi ketika ada kenikmatan indria,\\
 Maka mereka tunduk pada kesakitan dari pembunuhan dan ikatan.

Bagaikan obor rumput menyala\\
 Membakar seseorang yang memegangnya tanpa melepaskannya,\\
 Kenikmatan indria adalah seperti obor rumput,\\
 Yang membakar mereka yang tidak melepaskan.

Jangan lepaskan kebahagiaan berlimpah\\
 Demi kegembiraan kecil dari kenikmatan indria.\\
 Jangan menderita kesulitan kelak,\\
 Bagaikan ikan lele pada kail.

Kendalikan dirimu dengan seksama di antara kenikmatan-kenikmatan indria!\\
 Engkau bagaikan seekor anjing yang terikat pada rantai:\\
 Kenikmatan indria pasti akan melahap engkau\\
 Bagaikan kaum buangan yang lapar melahap anjing.

Karena terikat pada kenikmatan indria,\\
 Engkau mengalami kesakitan tanpa akhir,\\
 Bersama dengan banyak kesusahan batin:\\
 Lepaskanlah kenikmatan indria, itu tidak bertahan lama!

Ketika tanpa penuaan dapat ditemukan,\\
 Apakah gunanya kenikmatan indria yang di dalamnya terdapat usia tua?\\
 Semua kelahiran kembali di manapun\\
 Terikat pada kematian dan penyakit.

Ini adalah tanpa-penuaan, ini adalah tanpa-kematian\\
 Ini adalah tanpa-penuaan dan tanpa-kematian, keadaan tanpa dukacita!\\
 Bebas dari permusuhan, tidak menghambat,\\
 Tanpa cacat, tanpa ketakutan, tanpa kesusahan.

Tanpa-kematian ini telah direalisasikan oleh banyak orang;\\
 Bahkan sekarang itu dapat dicapai\\
 Oleh mereka yang dengan benar mengerahkan diri mereka;\\
 Tetapi itu tidak mungkin jika engkau tidak berusaha.''

Demikianlah Sumedhā berkata,\\
 Tanpa kesenangan pada hal-hal yang terkondisi.\\
 Menenangkan Anīkaratta,\\
 Sumedhā membuang rambutnya ke tanah.

Sambil berdiri, Anīkaratta\\
 Merangkapkan tangannya ke arah ayahnya dan memohon:\\
 ``Lepaskanlah Sumedhā, agar ia dapat meninggalkan keduniawian!\\
 Ia akan melihat kebenaran kebebasan.''

Dengan dilepaskan oleh ibu dan ayahnya,\\
 Ia meninggalkan keduniawian, takut pada kesedihan dan ketakutan.\\
 Sewaktu masih menjadi seorang calon bhikkhunī ia mencapai enam pengetahuan
langsung\\
 Bersama dengan buah tertinggi.

Padamnya sang putri\\
 sungguh mengagumkan dan menakjubkan;\\
 Di atas ranjang kematiannya, ia menyatakan\\
 Beberapa kehidupan lampaunya.

``Pada masa Buddha Ko\d{n}āgamana,\\
 Kami tiga sahabat memberikan persembahan\\
 Tempat tinggal yang baru dibangun\\
 Dalam vihara Sa\.{n}gha.

Sepuluh kali, seratus kali,\\
 Seribu kali, sepuluh ribu kali\\
 Kami terlahir kembali di antara para dewa,\\
 Apalagi di antara manusia.

\newpage

Kami perkasa di tengah-tengah para dewa,\\
 Apalah di tengah-tengah manusia!\\
 Aku adalah permaisuri dari seorang raja yang memiliki tujuh pusaka---\\
 Aku adalah pusaka istri.

Itu adalah penyebab, itu adalah asal-mula, itu adalah akar,\\
 Itu adalah penerimaan pengajaran;\\
 Pertemuan pertama yang memuncak dalam pemadaman\\
 Bagi seorang yang bersenang dalam ajaran.

Begitulah yang dikatakan oleh mereka yang berkeyakinan dalam kata-kata\\
 Dari seorang yang tak tertandingi dalam kebijaksanaan.\\
 Mereka dikecewakan oleh kelahiran kembali,\\
 Dan karena dikecewakan mereka menjadi bosan.'' 
\end{quote}
Demikianlah syair-syair ini dilafalkan oleh bhikkhunī senior Sumedhā.

Kelompok Panjang selesai.

Y pye

\newpage

\hidefromtoc

\chapter{Credits}
\begin{quote}
Tidak melakukan segala perbuatan buruk, \\ melakukan perbuatan baik,
\\membersihkan batin sendiri, \\ ini adalah ajaran para Buddha. 
\end{quote}
\begin{flushleft}
This book is a translation of Therīgāthā by Indra Anggara in 2020. 
\par\end{flushleft}

\begin{flushleft}
This volume was typeset and reviewed by members of Vihara Santi Graha,
Tanjung Redeb, Berau in March 2022 on International Women's Day. 
\par\end{flushleft}

\begin{flushleft}
This book would not have been possible without the work of Bhante
Sujato and his website https://suttacentral.net/ . We thank Bhante
Sujato for his efforts in making the Dhamma accessible and his work
in fighting patriarchy. 
\par\end{flushleft}

This book, like the translation made by Indra Anggara and related
works by Bhante Sujato's SuttaCentral is released into the public
domain, without copyright, and may be freely edited, copied, printed,
and shared without permission. 

\\2in

Computer files available online:

https://github.com/indodhamma/therigatha
\end{document}
